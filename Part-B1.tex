% Options for packages loaded elsewhere
\PassOptionsToPackage{unicode}{hyperref}
\PassOptionsToPackage{hyphens}{url}
\PassOptionsToPackage{dvipsnames,svgnames,x11names}{xcolor}
%
\documentclass[
  11pt,
  a4paperpaper,
  onecolumn]{article}

\usepackage{amsmath,amssymb}
\usepackage{lmodern}
\usepackage{iftex}
\ifPDFTeX
  \usepackage[T1]{fontenc}
  \usepackage[utf8]{inputenc}
  \usepackage{textcomp} % provide euro and other symbols
\else % if luatex or xetex
  \usepackage{unicode-math}
  \defaultfontfeatures{Scale=MatchLowercase}
  \defaultfontfeatures[\rmfamily]{Ligatures=TeX,Scale=1}
\fi
% Use upquote if available, for straight quotes in verbatim environments
\IfFileExists{upquote.sty}{\usepackage{upquote}}{}
\IfFileExists{microtype.sty}{% use microtype if available
  \usepackage[]{microtype}
  \UseMicrotypeSet[protrusion]{basicmath} % disable protrusion for tt fonts
}{}
\makeatletter
\@ifundefined{KOMAClassName}{% if non-KOMA class
  \IfFileExists{parskip.sty}{%
    \usepackage{parskip}
  }{% else
    \setlength{\parindent}{0pt}
    \setlength{\parskip}{6pt plus 2pt minus 1pt}}
}{% if KOMA class
  \KOMAoptions{parskip=half}}
\makeatother
\usepackage{xcolor}
\usepackage[left=2cm,top=1.5cm,right=2cm,bottom=1.5cm]{geometry}
\setlength{\emergencystretch}{3em} % prevent overfull lines
\setcounter{secnumdepth}{-\maxdimen} % remove section numbering
% Make \paragraph and \subparagraph free-standing
\ifx\paragraph\undefined\else
  \let\oldparagraph\paragraph
  \renewcommand{\paragraph}[1]{\oldparagraph{#1}\mbox{}}
\fi
\ifx\subparagraph\undefined\else
  \let\oldsubparagraph\subparagraph
  \renewcommand{\subparagraph}[1]{\oldsubparagraph{#1}\mbox{}}
\fi


\providecommand{\tightlist}{%
  \setlength{\itemsep}{0pt}\setlength{\parskip}{0pt}}\usepackage{longtable,booktabs,array}
\usepackage{calc} % for calculating minipage widths
% Correct order of tables after \paragraph or \subparagraph
\usepackage{etoolbox}
\makeatletter
\patchcmd\longtable{\par}{\if@noskipsec\mbox{}\fi\par}{}{}
\makeatother
% Allow footnotes in longtable head/foot
\IfFileExists{footnotehyper.sty}{\usepackage{footnotehyper}}{\usepackage{footnote}}
\makesavenoteenv{longtable}
\usepackage{graphicx}
\makeatletter
\def\maxwidth{\ifdim\Gin@nat@width>\linewidth\linewidth\else\Gin@nat@width\fi}
\def\maxheight{\ifdim\Gin@nat@height>\textheight\textheight\else\Gin@nat@height\fi}
\makeatother
% Scale images if necessary, so that they will not overflow the page
% margins by default, and it is still possible to overwrite the defaults
% using explicit options in \includegraphics[width, height, ...]{}
\setkeys{Gin}{width=\maxwidth,height=\maxheight,keepaspectratio}
% Set default figure placement to htbp
\makeatletter
\def\fps@figure{htbp}
\makeatother
\newlength{\cslhangindent}
\setlength{\cslhangindent}{1.5em}
\newlength{\csllabelwidth}
\setlength{\csllabelwidth}{3em}
\newlength{\cslentryspacingunit} % times entry-spacing
\setlength{\cslentryspacingunit}{\parskip}
\newenvironment{CSLReferences}[2] % #1 hanging-ident, #2 entry spacing
 {% don't indent paragraphs
  \setlength{\parindent}{0pt}
  % turn on hanging indent if param 1 is 1
  \ifodd #1
  \let\oldpar\par
  \def\par{\hangindent=\cslhangindent\oldpar}
  \fi
  % set entry spacing
  \setlength{\parskip}{#2\cslentryspacingunit}
 }%
 {}
\usepackage{calc}
\newcommand{\CSLBlock}[1]{#1\hfill\break}
\newcommand{\CSLLeftMargin}[1]{\parbox[t]{\csllabelwidth}{#1}}
\newcommand{\CSLRightInline}[1]{\parbox[t]{\linewidth - \csllabelwidth}{#1}\break}
\newcommand{\CSLIndent}[1]{\hspace{\cslhangindent}#1}

%----------------------------------------------------
% 	CONFIGURATION OF PARAMETERS
%----------------------------------------------------
% ----------------
%  FONTS AND TYPESETTING SETTINGS
% -----------------
\usepackage[english]{babel}
%\usepackage[bitstream-charter]{mathdesign}
%use\usepackage{times}
\usepackage{fontawesome5}
\usepackage{academicons}
\usepackage[notransparent]{svg}
\usepackage{tabu}
%\usepackage{coloremoji}
%\usepackage{emoji}

\usepackage{longtable}

%\usepackage[tracking=smallcaps]{microtype}
%\usepackage[bitstream-charter]{mathdesign}

% ----------------
%  STYLES OF THE CHAPTER, SECTION
% -----------------
%Options: Sonny, Lenny, Glenn, Conny, Rejne, Bjarne, Bjornstrup
%\usepackage[Bjornstrup]{fncychap}
\usepackage{marginnote}
\renewcommand*{\marginfont}{\small\sffamily}
\usepackage{wrapfig}
\usepackage{floatflt}
  
%---------------------------------
% Modification du model de UL
%\graphicspath{{./Figures/}} % chemin des figures


% -----------------
%  DEFINITION OF THE COLORS
% -----------------
%\usepackage{color} % Color
%\usepackage[usenames,dvipsnames,table]{xcolor}
%\usepackage{colortbl}

% -----------------
% 	MATHEMATHICAL SYMBOLS
% -----------------
%\usepackage{amssymb} %% The amssymb package provides various useful mathematical symbols
%\usepackage{amsthm} %% The amsthm package provides extended theorem environments
%\usepackage{amsmath}  % Paqute de Matematicas
%----------------------------------------------------------------------------------------

% -----------------
%  BIBLIOGRAPHY
% -----------------
%\usepackage  {natbib}
\usepackage{csquotes} %a utiliser si biblatex est utilisé
%\usepackage[style=numeric-comp, doi=false, backend=bibtex, isbn=false, url=false, sorting=none, language=english ]{biblatex}
%\addbibresource{library.bib}

%\usepackage[toc,page]{appendix}
\usepackage{booktabs}
% -----------------
%  VARIOUS PACKAGES FROM ME
% -----------------
%\usepackage{minitoc}
%\usepackage{booktabs} % Horizontal rules in tables
%\usepackage{float} % Required for tables and figures in the multi-column environment - they need to be placed in specific locations with the [H] (e.g. \begin{table}[H])
%\usepackage[lofdepth,lotdepth]{subfig} 
\usepackage{multirow} % Use multirows in the tables

%\usepackage {textcomp} % (Symbols Euros)
%\usepackage{nomencl} % nomenclature package
%\usepackage{soul}
%\renewcommand\thesection{\arabic{section}} %Redefinition of numeration of the document
%\usepackage{pdflscape} % hacer el documento lanscape en las hojas
%\usepackage{enumitem} % offers ready-made options for eliminating the space between items and paragraphs within the list (noitemsep) or all vertical spacing (nosep): 

\usepackage{enumitem}
\setlist[itemize]{noitemsep}

%\usepackage{afterpage} % For make blank pages
\usepackage{pdfpages} % For insert the title page.

\usepackage{mdframed}



% -----------------
%  VARIOUS PACKAGES FROM UNIVERSITE DE LORRAINE
% -----------------
%\usepackage{import}


%----------------------------------------------------------------------
%  PERSONALIZATION OF PACKAGES
%----------------------------------------------------------------------


% -----------------
%  DEFINITION OF COMMANDS BY ME
% -----------------
%\pagenumbering{Roman}






\usepackage{eurosym}

% To fix list things: 
\usepackage{enumitem}
\setitemize{noitemsep,topsep=0pt,parsep=0pt,partopsep=0pt,leftmargin=*}
\usepackage{amssymb}
\renewcommand{\labelitemi}{\tiny$\blacksquare$}

\usepackage{nopageno}




\usepackage{fancyhdr}
\pagestyle{fancy}
\renewcommand{\headrulewidth}{0pt} % Remove line at top
\setlength{\headheight}{14pt}

% Header
\lhead{\emph{Cruz Sanchez}} % Left of header
\chead{Part B1}
\rhead{DRAM Version 0.0.9}


% center of footer
%\fancyfoot[CO,CE]{And this is a fancy footer}
% page number on the left of even pages and right of odd pages
%\fancyfoot[LE,RO]{\thepage}
\cfoot{\thepage}

% 
% \newenvironment{itemize*}%
%   {\begin{itemize}%
%     \setlength{\itemsep}{0pt}%
%     \setlength{\parskip}{0pt}}%
%   {\end{itemize}}
% 

% 
\let\paragraph\oldparagraph
\let\subparagraph\oldsubparagraph
% 
\usepackage[compact]{titlesec}         % you need this package
\titlespacing{\subsubsection}{0pt}{3pt}{0pt} % this reduces space between (sub)sections to 0pt, for example
% \AtBeginDocument{%                     % this will reduce spaces between parts (above and below) of texts within a (sub)section to 0pt, for example - like between an 'eqnarray' and text
%   \setlength\abovedisplayskip{0pt}
%   \setlength\belowdisplayskip{0pt}}

\usepackage{indentfirst}
\setlength{\parindent}{15pt}% too much in my eyes delete this

\usepackage{lineno}
\makeatletter
\@ifpackageloaded{tcolorbox}{}{\usepackage[many]{tcolorbox}}
\@ifpackageloaded{fontawesome5}{}{\usepackage{fontawesome5}}
\definecolor{quarto-callout-color}{HTML}{909090}
\definecolor{quarto-callout-note-color}{HTML}{0758E5}
\definecolor{quarto-callout-important-color}{HTML}{CC1914}
\definecolor{quarto-callout-warning-color}{HTML}{EB9113}
\definecolor{quarto-callout-tip-color}{HTML}{00A047}
\definecolor{quarto-callout-caution-color}{HTML}{FC5300}
\definecolor{quarto-callout-color-frame}{HTML}{acacac}
\definecolor{quarto-callout-note-color-frame}{HTML}{4582ec}
\definecolor{quarto-callout-important-color-frame}{HTML}{d9534f}
\definecolor{quarto-callout-warning-color-frame}{HTML}{f0ad4e}
\definecolor{quarto-callout-tip-color-frame}{HTML}{02b875}
\definecolor{quarto-callout-caution-color-frame}{HTML}{fd7e14}
\makeatother
\makeatletter
\makeatother
\makeatletter
\makeatother
\makeatletter
\@ifpackageloaded{caption}{}{\usepackage{caption}}
\AtBeginDocument{%
\ifdefined\contentsname
  \renewcommand*\contentsname{Table of contents}
\else
  \newcommand\contentsname{Table of contents}
\fi
\ifdefined\listfigurename
  \renewcommand*\listfigurename{List of Figures}
\else
  \newcommand\listfigurename{List of Figures}
\fi
\ifdefined\listtablename
  \renewcommand*\listtablename{List of Tables}
\else
  \newcommand\listtablename{List of Tables}
\fi
\ifdefined\figurename
  \renewcommand*\figurename{Figure}
\else
  \newcommand\figurename{Figure}
\fi
\ifdefined\tablename
  \renewcommand*\tablename{Table}
\else
  \newcommand\tablename{Table}
\fi
}
\@ifpackageloaded{float}{}{\usepackage{float}}
\floatstyle{ruled}
\@ifundefined{c@chapter}{\newfloat{codelisting}{h}{lop}}{\newfloat{codelisting}{h}{lop}[chapter]}
\floatname{codelisting}{Listing}
\newcommand*\listoflistings{\listof{codelisting}{List of Listings}}
\makeatother
\makeatletter
\@ifpackageloaded{caption}{}{\usepackage{caption}}
\@ifpackageloaded{subcaption}{}{\usepackage{subcaption}}
\makeatother
\makeatletter
\@ifpackageloaded{tcolorbox}{}{\usepackage[many]{tcolorbox}}
\makeatother
\makeatletter
\@ifundefined{shadecolor}{\definecolor{shadecolor}{rgb}{.97, .97, .97}}
\makeatother
\makeatletter
\makeatother
\ifLuaTeX
  \usepackage{selnolig}  % disable illegal ligatures
\fi
\IfFileExists{bookmark.sty}{\usepackage{bookmark}}{\usepackage{hyperref}}
\IfFileExists{xurl.sty}{\usepackage{xurl}}{} % add URL line breaks if available
\urlstyle{same} % disable monospaced font for URLs
\hypersetup{
  colorlinks=true,
  linkcolor={blue},
  filecolor={Maroon},
  citecolor={Blue},
  urlcolor={blue},
  pdfcreator={LaTeX via pandoc}}

\author{}
\date{}

\begin{document}
\thispagestyle{fancy}
\begin{titlepage}

\begin{center}
   \large{\textbf{ERC Starting Grant 2023\\
   Research proposal [Part B1] }
   }
   \vspace{1cm}
   
   \LARGE{\textbf{My project title}}
   
   \vspace{1cm}
   
   \LARGE{\textbf{ACRO}}
   
   \vspace{1cm}
   
\normalsize   
\begin{itemize}
\item Principal investigator (PI): My Name
\item Host institution: My University
\item Full title: My project title
\item Proposal short name: ACRO
\item Proposal duration: 60 months
\end{itemize}
	
%%% SUMMARY
\noindent
\fbox{
\parbox{0.945\textwidth}{\normalsize
.
}
}

\vfill

\end{center}

\end{titlepage}

\ifdefined\Shaded\renewenvironment{Shaded}{\begin{tcolorbox}[interior hidden, sharp corners, borderline west={3pt}{0pt}{shadecolor}, boxrule=0pt, frame hidden, breakable, enhanced]}{\end{tcolorbox}}\fi

\hypertarget{section-a-extended-synopsis-of-the-scientific-proposal-max.-5-pages}{%
\section{Section a: Extended Synopsis of the scientific proposal {[}max.
5
pages{]}}\label{section-a-extended-synopsis-of-the-scientific-proposal-max.-5-pages}}

\begin{tcolorbox}[enhanced jigsaw, title=\textcolor{quarto-callout-note-color}{\faInfo}\hspace{0.5em}{DRAM in a nutshell}, colframe=quarto-callout-note-color-frame, colback=white, arc=.35mm, coltitle=black, toprule=.15mm, toptitle=1mm, opacityback=0, colbacktitle=quarto-callout-note-color!10!white, rightrule=.15mm, left=2mm, bottomrule=.15mm, bottomtitle=1mm, breakable, titlerule=0mm, leftrule=.75mm, opacitybacktitle=0.6]
The aim of Systemic Distributed Recycling for Additive Manufacturing
(SDRAM) is to establish a blueprint methodology for the implementation
of micro-value chains of distributed recycling at a urban territorial
level. We seek to the achievement of a three-level target: 1) Undertand
the establishment of a free-open source technical ecosystem that can be
printed, 2) to establish a set indicators to possible help
decision-makers and in the local implementation of these initiatives in
Europe/(America?), 3) \ldots..
\end{tcolorbox}

\hypertarget{the-state-of-the-art.}{%
\subsection{1. The State of the art.}\label{the-state-of-the-art.}}

\linenumbers

\hypertarget{arriving-to-the-limits-for-global-and-mass-manufacturing-paradigm}{%
\subsubsection{Arriving to the limits for global and mass manufacturing
paradigm}\label{arriving-to-the-limits-for-global-and-mass-manufacturing-paradigm}}

Plastic waste
contamination\textsuperscript{\protect\hyperlink{ref-de-la-torre2021}{1}},
climate change\textsuperscript{\protect\hyperlink{ref-stoddard2021}{2}},
biodiversity
loss\textsuperscript{\protect\hyperlink{ref-hermoso2022}{3}} are majors
markups of what is recently disscused as the Anthropocene
era\textsuperscript{\protect\hyperlink{ref-steffen2018}{4},\protect\hyperlink{ref-steffen2011}{5}}.
The anthropocene frames the humans not only as biological but as
geological force acknowledging the new status of humanity given the
different indicators in the natural ecosystems that are impacting the
stability of the earth system. The globalized mass manufacturing
paradigm have played a major role not only as motor for the economic
development, but also the transgression of the planetary
boundaries\textsuperscript{\protect\hyperlink{ref-ONeill2018}{6}--\protect\hyperlink{ref-Rockstrom2009}{8}}.
The mass manufacturing socio-technical systems is understood as a deep
transition\textsuperscript{\protect\hyperlink{ref-kanger2022}{9}}.
Manufacturing systems requires materials as well as human and physical
capital to produce goods. The co-evolution of single unit productions
systems, interconnected systems, and industrial modernity have been
gradually intensified various forms of environmental
degradation\textsuperscript{\protect\hyperlink{ref-ref}{\textbf{ref?}}}.
This co-evolution remains not to solve recurring issues of social
inequality in connection to unequal access to healthcare, energy, water,
food, mobility, security, finance, education, and
communication\textsuperscript{\protect\hyperlink{ref-ref}{\textbf{ref?}}}.
Even if the importance of manufacturing as the heart of an economy has
not changed, the way of producing goods and the setup of the location
start to change dramatically. The circular economy concept entry in the
policy\textsuperscript{\protect\hyperlink{ref-EC2015}{10}},
industrial\textsuperscript{\protect\hyperlink{ref-EllenMacArthurFoundation2015}{11}}
and
scientific\textsuperscript{\protect\hyperlink{ref-nobre2021}{12}--\protect\hyperlink{ref-Schoggl2020}{14}}
arenas as an umbrella concept, but also as a contested
one\textsuperscript{\protect\hyperlink{ref-CalistoFriant2020}{15}--\protect\hyperlink{ref-corvellec2021}{17}},
aiming to change the societal conciousness that the ecological systems
have nearly endless capacity to provide resources and adsorb wastes.
Engineering science needs to integrate that the
externalities\textsuperscript{\protect\hyperlink{ref-zhen2021}{18}} of
human activites' impacts on the earth systems since the fuzzy front-end
phase of the innovation process.

\begin{itemize}
\tightlist
\item
  \(\color{red}{\text{Paragraph à developper sur les systemes de Manufactures}}\)
\end{itemize}

\hypertarget{major-long-vision-circular-and-production}{%
\subsubsection{Major long vision: Circular and
production}\label{major-long-vision-circular-and-production}}

Today, a major societal issue rely on how to conceived socio-technical
`circular units' for manufacturing that integrates values of
sobriety\textsuperscript{\protect\hyperlink{ref-ref}{\textbf{ref?}}},
resilience\textsuperscript{\protect\hyperlink{ref-touriki2021}{19},\protect\hyperlink{ref-VanFan2019}{20}},
adaptability\textsuperscript{\protect\hyperlink{ref-weichhart2021}{21}}
and evolutive in urban settlements. The reuse, repairing, recycling
approaches will need to converge in a post-growth economy context
considering the societal issues of resource scarcity and waste
accumulation in the urban
settlements\textsuperscript{\protect\hyperlink{ref-kallis2018}{22},\protect\hyperlink{ref-savini2021}{23}}.
Indeed, today the establishment of these socio-technical systems need to
include all ecosystem externalitites and the carrying capacity of the
ecosystem to claim to
sustainability\textsuperscript{\protect\hyperlink{ref-Bakshi2018}{24},\protect\hyperlink{ref-Bakshi2019a}{25}}.
The trend is reinforced by the fact that by 2050, it is expected that
about 70\% of the world's population will live in urban
settlements\textsuperscript{\protect\hyperlink{ref-savini2021}{23}}.
Urban cities will be responsible for non-negligible environmental
impact\textsuperscript{\protect\hyperlink{ref-Zheng2020}{26},\protect\hyperlink{ref-Sodiq2019}{27}},
producing about 50\% of global waste, and 75\% of greenhouse gas
emissions which affects the sustainability of
cities\textsuperscript{\protect\hyperlink{ref-schraven2021}{28}} and the
quality of city
life\textsuperscript{\protect\hyperlink{ref-Riffat2016}{29}}.

\hypertarget{open-source-and-digital-commons-for-design-global-manufacturing-local}{%
\subsubsection{Open source and digital commons for `Design global /
Manufacturing
local'}\label{open-source-and-digital-commons-for-design-global-manufacturing-local}}

As an alternative of globalized manufacturing values chains, a major
trend in the development of production systems seeks to establish an
urban production
model\textsuperscript{\protect\hyperlink{ref-Herrmann2020}{30},\protect\hyperlink{ref-juraschek2022}{31}}
with decentralized and distributed
characteristics\textsuperscript{\protect\hyperlink{ref-priavolou2022}{32},\protect\hyperlink{ref-cerdas2017}{33}}.
Aiming at a \emph{`design global / manufacturing
local'}\textsuperscript{\protect\hyperlink{ref-Kostakis2018}{34}} seems
a
proto-industrialization\textsuperscript{\protect\hyperlink{ref-sabel1985}{35}}
transition that is taking place in urban settlements that could a major
impact in the next short future. The Open Source Appropriate Technology
(OSAT)\textsuperscript{\protect\hyperlink{ref-Pearce2010}{36}} and
peer-to-peer
(P2P)\textsuperscript{\protect\hyperlink{ref-Kostakis2013}{37}}
approaches have been seen potential drivers to propose an alternative
globalisation manufacturing
paradigm\textsuperscript{\protect\hyperlink{ref-Heikkinen2020a}{38}}.
The open source (OS) approach has become well-established to provide
improved product innovation over proprietary product
development\textsuperscript{\protect\hyperlink{ref-dibona1999}{39}--\protect\hyperlink{ref-deek2007}{42}}.
The evidence is most mature for software development because free and
open source software (FOSS) provides: i) diversification and open
innovation\textsuperscript{\protect\hyperlink{ref-colombo2014}{43}--\protect\hyperlink{ref-alexy2013}{45}},
ii) cumulative
innovation\textsuperscript{\protect\hyperlink{ref-boudreau2016}{46}},
iii) development
efficiency\textsuperscript{\protect\hyperlink{ref-hienerth2014}{47}},
iv) organizational
innovation\textsuperscript{\protect\hyperlink{ref-alexy2013}{45}}, v)
higher technical quality of
code\textsuperscript{\protect\hyperlink{ref-soderberg2015}{48}}, vi)
encourages
creativity\textsuperscript{\protect\hyperlink{ref-martinez2015}{49}} and
vii) perhaps most importantly, it avoids redundant
work\textsuperscript{\protect\hyperlink{ref-Ardal2016}{50}}. The OS
approach is now also gaining traction in free and open source hardware
(FOSH)\textsuperscript{\protect\hyperlink{ref-thompson2011}{51}--\protect\hyperlink{ref-li2018}{55}}
and appears to be roughly 15 years behind FOSS in development and
adoption\textsuperscript{\protect\hyperlink{ref-pearce2018}{56}}. One of
the primary drivers, is that all forms of free and open source
technology software and hardware (FOSS and FOSH) can provide a
substantial cost
savings\textsuperscript{\protect\hyperlink{ref-petch2014}{57}--\protect\hyperlink{ref-wittbrodt2013}{60}}.
The open source additive manufacturing technology, also know as 3D
printing, have played a major role in the idea of democratization of
manufacturing
means\textsuperscript{\protect\hyperlink{ref-Beltagui2020}{61}}.
Thousands of open-source products are shared by the global community
from consumer goods to
scientific\textsuperscript{\protect\hyperlink{ref-Pearce2020a}{62}} and
medical
equipment\textsuperscript{\protect\hyperlink{ref-Pearce2020a}{62},\protect\hyperlink{ref-He2014}{63}}.
This model has been proven to be effective for emergency manufacturing
during the COVID-19
pandemic\textsuperscript{\protect\hyperlink{ref-Pearce2020a}{62},\protect\hyperlink{ref-tan2021}{64}}.
This is a driver communities to fabricate their own products for less
than the price of purchasing them. In that sense, the concept of urban
factory is evolving as a disruptive approach and is the materialization
of this manufacturing paradigm. The urban factory is defined as
``\emph{a factory located in an urban environment that is actively
utilizing the unique characteristics of its surroundings}''. It creates
products with a focus on the local market and allows customer
involvement during value
creation\textsuperscript{\protect\hyperlink{ref-Herrmann2020}{30},\protect\hyperlink{ref-Ijassi2022}{65}}.

\hypertarget{distributed-recycling-for-additive-manufacturing-a-promising-inclusion}{%
\subsubsection{Distributed recycling for additive manufacturing: a
promising
inclusion}\label{distributed-recycling-for-additive-manufacturing-a-promising-inclusion}}

Since 2014, I have been working on the validation of the open-source 3D
printing,
filament-\textsuperscript{\protect\hyperlink{ref-CruzSanchez2014}{66}}
and
pellet-based\textsuperscript{\protect\hyperlink{ref-Arthur2020}{67}}, as
a robust manufacturing system, but also as a potential enabler of the
mechanical
recycling\textsuperscript{\protect\hyperlink{ref-Cruz2015}{68}--\protect\hyperlink{ref-lopez2022}{70}}
of plastic waste feedstock. Likewise, I have been working on the design
of the pertinent closed-loop supply
chain\textsuperscript{\protect\hyperlink{ref-Pavlo2018}{71},\protect\hyperlink{ref-Santander2020}{72}},
considering some of the sustainability
indicators\textsuperscript{\protect\hyperlink{ref-Santander2022}{73}}
needed in the based on the literature. In a recent
paper\textsuperscript{\protect\hyperlink{ref-CruzSanchez2020}{74}}, I
could highthligh a great interest by the scientific community of thi
topic which is called distributed recycling for additive manufacturing
(DRAM). DRAM (See Figure~\ref{fig-DRAM}) is a breakthrough promise in
the constitution of a micro-circular industry units to validate the
technical feasibility, and several technological pathways are maturing
to allow individuals to recycle waste plastic directly by 3D-printing it
into valuable products.

\begin{figure}

{\centering \includegraphics[width=0.9\textwidth,height=\textheight]{Figures/SDRAM-00.png}

}

\caption{\label{fig-DRAM}Distributed recycling via additive
manufacturing. Source}

\end{figure}

To appreciate the ground-breaking scientific nature of this idea, let me
state that the most adopted form of additive manufacturing is fused
filament fabrication (FFF), which is a material extrusion process
{[}@{]}. DRAM starts with waste plastic that is produced everywhere from
packaging to broken products (\emph{Recovery (I)}). It is washed, dried
and then ground or cut into particles using a waste plastic granulator
or office shredder (\emph{Preparation (II)}). The raw material for FFF
can be manufactured economically using distributed means with a waste
plastic extruder (often called a
``recyclebot'')\textsuperscript{\protect\hyperlink{ref-Baechler2013}{75}}
for mono or composite materials (\emph{Compounding (II) and Feedstock
(IV)}). Filament made with a recyclebot costs less than 10 cents per kg,
whereas commercial filament costs \$20/kg or more. This can produce
valuable products at remarkably low costs. For example, using a
recyclebot/3D-printer combination can produce over 300 units (e.g.,
camera lens hoods) for the price of one such item listed on Amazon.com.
Fused granular fabrication is a recent experimental approach enabling
the printing process directly from
pellets\textsuperscript{\protect\hyperlink{ref-JustinoNetto2021}{76},\protect\hyperlink{ref-netto2022}{77}},
which reduces the degradation cycles of the plastic. For this process, I
worked in the desktop
format\textsuperscript{\protect\hyperlink{ref-Arthur2020}{67}}, but it
seems that this technology could further expand the boundaries of
additive manufacturing and eventually
recycling\textsuperscript{\protect\hyperlink{ref-billah2021}{78}--\protect\hyperlink{ref-Byard2019}{80}}
for larger
object\textsuperscript{\protect\hyperlink{ref-petsiuk2022}{81}}.
Distributed recycling fits into the circular economy
paradigm\textsuperscript{\protect\hyperlink{ref-Zhong2018}{82}--\protect\hyperlink{ref-Despeisse2016}{84}},
as it eliminates most embodied energy and pollution from transportation
between processing steps. Also, it decreases the embodied energy of
filament by 90\% compared to traditional centralized filament
manufacturing using fossil fuels as
inputs\textsuperscript{\protect\hyperlink{ref-Kreiger2013}{85}--\protect\hyperlink{ref-Horta2017}{87}}.
Additionaly, open-source investment should result in an extremely high
return on investment
(ROI)\textsuperscript{\protect\hyperlink{ref-Pearce2020a}{62}}. This
makes distributed recycling environmentally superior to other methods of
plastic recycling systems.

However, I realized that the global system maturity is ambiguous given
that not all the value chain for the implementation of a
community-driven of plastic recycling are
matured\textsuperscript{\protect\hyperlink{ref-CruzSanchez2020}{74}}.
Major efforts in the scientific literature have been only concentrated
in the materials and technical validation.\\
However, the system validation remains to be difficult to implement.
More important, the analysis of the holistic impact that this process
can have in the context of a city remains not well understood. In the
framework of a EUH2020 project called INEDIT\footnote{See
  https://cordis.europa.eu/project/id/869952}, I have been leading the
implementation of the \emph{Green Fablab} demostrator inside the third
place called Octroi-Nancy Association \footnote{See
  https://www.octroi-nancy.fr/} since November 2021\footnote{This
  demostrator found retard because of the pandemic situation.}. INEDIT
project aims to create an ecosystem to transform the
\emph{Do-It-Yourself} practices largely documented in
FabLabs/Hacker/Maker spaces into a professional approach called
Do-It-Together to capitalise on the knowledge, creativity and ideas of
design and engineering. The Green Fablab is a distributed recycling
demostrator that that use living lab
approach\textsuperscript{\protect\hyperlink{ref-tyl2021}{88},\protect\hyperlink{ref-compagnucci2020a}{89}}
to experiment in real conditions with citizens, final users and large
general public. This experiment is enframed as a design for
sustainability at a socio-technical system
level\textsuperscript{\protect\hyperlink{ref-Ceschin2016}{90}}. We have
collected and recycling around 100kg of plastic waste for the
pedagogical and architectural uses given the fact that we are connected
with a creative ecosystem of designers and makers participatin in the
Octroi-Nancy projet. This hands-on experience confirms the literature
that a recycled resources industry (RRI) is starting to conceived inside
the cities\textsuperscript{\protect\hyperlink{ref-wang2019b}{91}}. RRI
is seen as driver consists of a series of activities related to recycled
resources -- e.g., recycling, refining, remanufacturing, etc. --
aspiring to mitigate the negative externality caused by the linear
economy . The sustainable development of the RRI has thus been
highlighted on many countries' agendas to promote the circular
society\textsuperscript{\protect\hyperlink{ref-leipold2021}{92}--\protect\hyperlink{ref-jaeger-erben2021a}{94}},
as well as the goals of carbon peak and carbon neutralization. In the
case of plastic waste, the main difficulty remains to make affordable
the use of new secondary material applicability by the
industry\textsuperscript{\protect\hyperlink{ref-klotz2022}{95}}, but
more profoundly, how these socio-technical experiments will interact
with the urban planning and polycimaking to make concrete the ambition
of circular economy inside the urban and regional settlements.

\hypertarget{ambition-objectives}{%
\subsection{2. Ambition \& objectives}\label{ambition-objectives}}

The material
rarefaction\textsuperscript{\protect\hyperlink{ref-hultman2021}{96}},
the ecological integration in the fuzzy-front end design of
manufacturing
systems\textsuperscript{\protect\hyperlink{ref-Bakshi2019a}{25},\protect\hyperlink{ref-Bakshi2015}{97},\protect\hyperlink{ref-Saladini2018}{98}}
and the resilience of production
systems\textsuperscript{\protect\hyperlink{ref-xu2021e}{99}} remains a
systemic problem and it calls for pushing forward the boundaries of
knowledge in the fuzzy front-end design phases of socio-technical
manufacturing configurations. There is a urgent necessity to better
understand how to design,
orchestrate\textsuperscript{\protect\hyperlink{ref-ritala2022}{100}} and
evaluate the socio-technical circular demonstrators at urban levels to
unleash a sustainability transition towards appropriate and inclusive
micro-manufacturing and recycling values chains inspired on the
\emph{``Design Global / Manufacturing local'' principles}. By exploring
the case of Green Fablab At Octroi Nancy, \textbf{the purpose of SDRAM
project is create a systemic methodological blueprint approach to fully
expand the frontiers of the design socio-technical manufacturing systems
as a sustainable transitions in urban settlements.} To do so, the SDRAM
project aims to deep understanding of the three major layers and the
boundary objects between them:

\begin{enumerate}
\def\labelenumi{\arabic{enumi}.}
\tightlist
\item
  Urban space in the lens of the urban manufacturing opportunity and
  material rarefaction.
\item
  Design for a technodiversity baseline based on open source appropriate
  technologies (OSAT) for distributed recycling, and
\item
  Pluralistic (e)valuation of socio-technical alternatives to mass
  production in the frame of a urban sustainability transition.
\end{enumerate}

\hypertarget{manufacturing-and-an-urban-priority-for-resilience-and-agility.}{%
\paragraph{\texorpdfstring{\emph{Manufacturing and an urban priority for
resilience and
agility}.}{Manufacturing and an urban priority for resilience and agility.}}\label{manufacturing-and-an-urban-priority-for-resilience-and-agility.}}

The significance and main challengue of sustainable urban production
lies in the bridging of disciplinary boundary of urban and manufacturing
systems fields\textsuperscript{\protect\hyperlink{ref-Tsui2020}{101}}.
One major drawback is the lack of holistic and shared framework to
connect the urban and manufacturing development. There is an opportunity
to create a City-Factory-Product
nexus\textsuperscript{\protect\hyperlink{ref-herrmann2019}{102},\protect\hyperlink{ref-williams2019}{103}}
understanding that aims to be adaptable,
resilient\textsuperscript{\protect\hyperlink{ref-Shabbir2021}{104},@
\protect\hyperlink{ref-mubarik2021}{105}} and considering the carrying
capacity of the urban ecosystem.

\hypertarget{the-open-source-appropriate-technology-osat-as-alternative.}{%
\paragraph{\texorpdfstring{\emph{The open-source appropriate technology
(OSAT) as
alternative}.}{The open-source appropriate technology (OSAT) as alternative.}}\label{the-open-source-appropriate-technology-osat-as-alternative.}}

The OSAT relies on small-scale, economically affordable, decentralised,
energy-efficient, environmentally sound and easily utilized by local
communities to meet their
needs\textsuperscript{\protect\hyperlink{ref-Pearce2012b}{106}}. This
approach have been valuable for scientific equipement to reduce the cost
with equal of
quality\textsuperscript{\protect\hyperlink{ref-Pearce2014k}{107},\protect\hyperlink{ref-Pearce2016}{108}},
and having implication in national
level\textsuperscript{\protect\hyperlink{ref-Heikkinen2020a}{38},\protect\hyperlink{ref-pearce2022a}{109}}.
Therefore, a OSAT technodiversity is a breakthrough to possible open up
the valorization of material loops inside urban settlements fostering
the creation of urban closed-loop supply chains. The establishment of
development of a technological open source maturity level focalised on
the distributed recycling is part of the technical blueprint.

\hypertarget{pluralistic-evaluation-for-emerging-industrial-micro-values-chains-that-integrate-ecosystem-characteristics.}{%
\paragraph{\texorpdfstring{\emph{Pluralistic (e)valuation for emerging
industrial micro-values chains that integrate ecosystem
characteristics.}}{Pluralistic (e)valuation for emerging industrial micro-values chains that integrate ecosystem characteristics.}}\label{pluralistic-evaluation-for-emerging-industrial-micro-values-chains-that-integrate-ecosystem-characteristics.}}

Reconciling urban development and industrial development is not an easy
task (wicked problem). Thus, the type of information that
decision-makers take into account is relevant at the moment to put in
place industrial systems. From systemic design thinking and ecological
economics
fileds\textsuperscript{\protect\hyperlink{ref-kish2021}{110}@,\protect\hyperlink{ref-economics2021}{111}},
it is needed to new forms of (e)valuation beyond the
economics\textsuperscript{\protect\hyperlink{ref-gunton2022}{112}} to
identify major feedbacks in the strategic, the tactical and the
operational decisional levels. The integration of ecological
aspects\textsuperscript{\protect\hyperlink{ref-kennedy2022}{113}} in the
decision-making seems not evident given the complexity to define the
boundaries and interactions of industrial and ecological systems.
However, it is urgent to expand the boundaries for engineering design
from the lowest molecular- / process-level, to the higher levels of
value chains, ecosystems and the
planet\textsuperscript{\protect\hyperlink{ref-Martinez-Hernandez2017}{114},\protect\hyperlink{ref-kurtz2021}{115}}.
We need to integrate ecological carrying capacity since the fuzzy front
end phase of an industrial systems.

The ambition of this project is to open up the possibilities of a new
field of socio-technical design of distributed and circular urban
production systems to the scientific community.

\hypertarget{a-challenging-task-for-a-systemic-blueprint}{%
\subsubsection{A challenging task for a systemic
blueprint}\label{a-challenging-task-for-a-systemic-blueprint}}

The major gap that currently prevents from exploring the potential of
alternative distributed and circular manufacturing relies on a knowledge
gap in terms of the maturity in the connection between the
unit-facility-urban levels including the respective boundary
objects\textsuperscript{\protect\hyperlink{ref-Abson2014}{116}} that
needs to be considered between the layers. From a design for
sustainability\textsuperscript{\protect\hyperlink{ref-Ceschin2016}{90},\protect\hyperlink{ref-SousaRocha2019}{117}}
perspective, this implies the aid-decision tools to help makers,
practitioners and decision-makers in the implementation phase
considering the technosphere (molecule, material, process unit) but also
the also to the ecosystem impact. Therefore as a systemic blueprint, I
aim to make linkage of the micro-meso-macro levels of the technical,
system and valuation layers embeeded in a urban spatio-temporal context
(See Figure network)

\begin{figure}

{\centering \includegraphics[width=0.5\textwidth,height=\textheight]{Figures/Levels.jpeg}

}

\caption{\label{fig-DRAM}Grafic a faire baseee sur celui-la Source XX}

\end{figure}

Rethinking the design of efficient and effective production system under
the perspective of small and modular machines in combination with the
means provided by rapidly increasing digitalization can support the
development of sustainable and competitive urban production systems.
Urban production systems can be developed in a way that is sustainable
and competitive by rethinking the design of efficient and effective
production systems from the perspective of tiny and modular machines in
conjunction with the tools afforded by rapidly rising OSAT.

\hypertarget{the-methodology}{%
\subsection{3. The Methodology}\label{the-methodology}}

\begin{wrapfigure}[13]{r}[0pt]{0.35\textwidth}
\centering
    \includegraphics[width=\linewidth]{Figures/WPs.pdf}
    \caption{Methodology}
    \label{fig:WPs}
\end{wrapfigure}

SDRAM implement a methodology made of four working packages (WP), as
illustrated in Fig. \ref{fig:WPs}. The aim of WP1 is to set a literature
baseline for an integrative and critical analysis of urban territory in
the frame of micro-value chains for local recycling loops. This working
package gives the insights for the WP2, and WP3, which are key of the
project. The WP2 seeks to consolidate systematize a design process for
OSAT for a complete distributed recycling process establishing an unit
maturity level index for each, but more important, a system maturity
level for the integration in a urban ecosystem. The main goal is to
establish a complete OSAT ecosystems to valorize the waste niches
opportunities identified in WP1.\\
The WP3 aims to identify a pluralistic (e)valuation framework for the
urban closed-loop system network integrating three essential issues:
sustainability, resiliency, and agility into a circular economy praxis.
Finally, WP4 is dedicated to the experimentation of the several products
case studies of the urban circular manufacturing taking into at case
studies the implementation of the Green Fablab Project at the third
place of OK3 at Nancy-France. The object is to replicate this analysis
in other territories such Chile, in collaboration with Prof.~Pavlo
Santander, and in Canada with collaboration of Joshua Pearce. Work
packages are synthetically detailed hereinafter.

\hypertarget{wp-1-theoretical-baseline-on-urban-value-chains}{%
\subsubsection{WP 1: Theoretical baseline on urban value
chains}\label{wp-1-theoretical-baseline-on-urban-value-chains}}

WP1 aims at developing a integral methodology to diagnose, quantify and
evaluate the potential urban value chains for distributed recycling
loops on a territory considering the ecological priorities of the
territory. The achievement to SDRAM target relies the urban spatial
analysis and stakeholders characteristics as an entry point of the
design of the socio-technical system mapping two major outputs: 1.1) The
first output aims to highlights: (a) the identification of the
priorities in terms of ecosystems services of the territory at the urban
planning level, and how the plastic waste affects them. (b) the
evaluation (technical, economic and environmental) of the current waste
management system to identify the ' of the limits of the loop chains ,
existing plastic `gaps' that distributed recycling approach can fill,
and (c), a stakeholder characterization analysis needs (e.g.~sorting
centres, recycling centres, schools). Then in 1.2), the second output
aims to close the existing data
gaps\textsuperscript{\protect\hyperlink{ref-Bianchi2020}{118}} in terms
of secondary material availability at the urban level considering its
complexity level of revalorization. The goal is to couple
\emph{\{territory x material\}} together as a material flow quantitative
analysis to assess the potential to material for a closed-loop supply
chain. This is particularly relevant in the context of plastic products
where governments worldwide are placing ambitious circularity targets
due to the accumulation. The priority is to reveal a list of `suitable'
secondary plastic materials wastes at the urban level that today are not
fully understood and valorized. This analysis will be carried out at
least every year, and if possible more frequently to see if there is a
change or seasonality in the composition of this untreated waste.

\hypertarget{wp-2-maturity-level-and-technodiverstity-level-of-the-open-source-appropritte-technology}{%
\subsubsection{WP 2: Maturity level and technodiverstity level of the
open source appropritte
technology}\label{wp-2-maturity-level-and-technodiverstity-level-of-the-open-source-appropritte-technology}}

The WP2 will be focused on the unit- and facility-level to better
understand how OSAT can be implemented in urban micro-recycling systems.
The main purpose of this task is to leverage a resilient
manufacturing\textsuperscript{\protect\hyperlink{ref-xu2021e}{99},\protect\hyperlink{ref-zhang2011}{119}}
under the logic of Design Global/Manufacture Local robustness. To do so,
three major tasks are seen:

2.1) definition of a scientific literature and critical analysis on the
adoption\textsuperscript{\protect\hyperlink{ref-reinauer2021}{120}} and
barriers of the open-source appropriate technologies with particular
focus on distributed recycling considering the modularity
types\textsuperscript{\protect\hyperlink{ref-gavras2021}{121}}, gaps in
the hardware development and .\\
2.2) Mapping of new/adapted practices and tools that would be needed to
support local manufacturers and local decision makers to navigate and
overcome the challenges of distributed recycling manufacturing. 2.3)
Identification a system maturity level that enable the constitution of
urban closed-loop supply chain . \ldots{}

\hypertarget{wp-3-pluralistic-evaluation-of-distributed-recycling-systems}{%
\subsubsection{WP 3: Pluralistic (e)valuation of distributed recycling
systems}\label{wp-3-pluralistic-evaluation-of-distributed-recycling-systems}}

In parallel of WP2, the WP3 aims to consolidate aid-decision tool to
reveal and better understand under which conditions these distributed
recycling/manufacturing urban chains are pertinent for the local
territory. This tool describe and characterize the new value chain to
include new form of pluralism
valuation\textsuperscript{\protect\hyperlink{ref-gunton2022}{112}} and
techno-ecological
interactions\textsuperscript{\protect\hyperlink{ref-Saladini2018}{98},\protect\hyperlink{ref-Liu2020c}{122},\protect\hyperlink{ref-Liu2019g}{123}}.
More important to avoid Jevons
paradox\textsuperscript{\protect\hyperlink{ref-giampietro2018}{124}}, it
is determine the scale of action considering the technical maturity,
economic viability and environmental respect of the ecosystem services.
In (4.1), one strategical point in sustainability relies on explicitly
account for their demand and supply of of ecosystem goods and services
framework given by the micro-value
chains\textsuperscript{\protect\hyperlink{ref-Diwekar2021}{125}}. then
(4.2), the main aim is to reveal the components and the structure of the
urban circular networks to the combining Material Flow
Analysis\textsuperscript{\protect\hyperlink{ref-saidani2021}{126}},
System
Dynamics\textsuperscript{\protect\hyperlink{ref-kuo2021}{127}--\protect\hyperlink{ref-perez-perez2021}{131}}
and Circularity
Indicators\textsuperscript{\protect\hyperlink{ref-saidani2019}{132}}.

\hypertarget{wp-4-experimentation-and-deployment-in-function-of-the-local-territory}{%
\subsubsection{WP 4: Experimentation and deployment in function of the
local
territory}\label{wp-4-experimentation-and-deployment-in-function-of-the-local-territory}}

The WP4 aims to consolidate a starting point for a longitudinal
study\textsuperscript{\protect\hyperlink{ref-langley2013}{133}} to
evaluate of the implementation these distributed recycling strategies at
a urban territorial level. WP4 is devoted to the iteration and
evaluation of the urban production networks to deep understand the
evolution. 4.1) Several case studies of distributed fabrication /
recycling will be documented and developed in complement with a
comparative and contextualized Life Cycle Assessment (LCA) of the new
secondary AM material compared to actual materials. 4.2) A strategic
roadmap will be a major delivered to understand the possible evolution
of

To pass from ecodesign to an operation design for sustainability
approach, this WP4 will be based ten different models at operational,
tactical, and strategical
levels\textsuperscript{\protect\hyperlink{ref-SousaRocha2019}{117}}.

\hypertarget{conceptual-risk-and-fesability-assessment}{%
\subsection{3. Conceptual risk and fesability
assessment}\label{conceptual-risk-and-fesability-assessment}}

SDRAM is a high operation and conceptual-risk project mainly because the
integration of multiples disciplines in a one basis framework need to
establish boundary object to have a coherent framework.

\small

\begin{longtable}[]{@{}
  >{\raggedright\arraybackslash}p{(\columnwidth - 10\tabcolsep) * \real{0.0233}}
  >{\raggedright\arraybackslash}p{(\columnwidth - 10\tabcolsep) * \real{0.2744}}
  >{\raggedright\arraybackslash}p{(\columnwidth - 10\tabcolsep) * \real{0.1256}}
  >{\raggedright\arraybackslash}p{(\columnwidth - 10\tabcolsep) * \real{0.0977}}
  >{\raggedright\arraybackslash}p{(\columnwidth - 10\tabcolsep) * \real{0.0419}}
  >{\raggedright\arraybackslash}p{(\columnwidth - 10\tabcolsep) * \real{0.4372}}@{}}
\caption{ss}\tabularnewline
\toprule()
\begin{minipage}[b]{\linewidth}\raggedright
ID
\end{minipage} & \begin{minipage}[b]{\linewidth}\raggedright
Risk items
\end{minipage} & \begin{minipage}[b]{\linewidth}\raggedright
Effect of the risk
\end{minipage} & \begin{minipage}[b]{\linewidth}\raggedright
Causes of the risk
\end{minipage} & \begin{minipage}[b]{\linewidth}\raggedright
Grade
\end{minipage} & \begin{minipage}[b]{\linewidth}\raggedright
Actions to minimize the risk
\end{minipage} \\
\midrule()
\endfirsthead
\toprule()
\begin{minipage}[b]{\linewidth}\raggedright
ID
\end{minipage} & \begin{minipage}[b]{\linewidth}\raggedright
Risk items
\end{minipage} & \begin{minipage}[b]{\linewidth}\raggedright
Effect of the risk
\end{minipage} & \begin{minipage}[b]{\linewidth}\raggedright
Causes of the risk
\end{minipage} & \begin{minipage}[b]{\linewidth}\raggedright
Grade
\end{minipage} & \begin{minipage}[b]{\linewidth}\raggedright
Actions to minimize the risk
\end{minipage} \\
\midrule()
\endhead
1 & Difficulty to data access to local territorial diagnosis &
Constraint to define WP1 & & Middle & There have been pre-exists between
the partners and these territories and recycling actors. \\
2 & & & & & \\
3 & & & & & \\
\bottomrule()
\end{longtable}

\begin{longtable}[]{@{}
  >{\raggedright\arraybackslash}p{(\columnwidth - 4\tabcolsep) * \real{0.0833}}
  >{\raggedright\arraybackslash}p{(\columnwidth - 4\tabcolsep) * \real{0.4583}}
  >{\raggedright\arraybackslash}p{(\columnwidth - 4\tabcolsep) * \real{0.4583}}@{}}
\caption{Feasible challengues in the methodology}\tabularnewline
\toprule()
\begin{minipage}[b]{\linewidth}\raggedright
ID
\end{minipage} & \begin{minipage}[b]{\linewidth}\raggedright
Main challengues
\end{minipage} & \begin{minipage}[b]{\linewidth}\raggedright
Feasibility
\end{minipage} \\
\midrule()
\endfirsthead
\toprule()
\begin{minipage}[b]{\linewidth}\raggedright
ID
\end{minipage} & \begin{minipage}[b]{\linewidth}\raggedright
Main challengues
\end{minipage} & \begin{minipage}[b]{\linewidth}\raggedright
Feasibility
\end{minipage} \\
\midrule()
\endhead
1 & Theoretical baseline on urban value chains & \\
2 & Maturity level and technodiverstity level of the open source
appropritte technology & \\
3 & Pluralism (e)valuation of the distributed recycling systems & \\
4 & & \\
\bottomrule()
\end{longtable}

\normalsize

\hypertarget{an-impact-project}{%
\subsection{4. An Impact project}\label{an-impact-project}}

\begin{itemize}
\item
  \textbf{Main scientific impacts.} (1) the breakthrough understating of
  the implementation and evaluation of the design of sustainability of
  socio-technical systems
\item
  \textbf{Main societal impacts.} If the expected modeling are
  confirmed, the outcome of this pproject will allow urban and technical
  desicion-makers the implementation of local recycling circuits of
  available plastic waste by means of small, ro distribed recycling
  socio-technical units.
\end{itemize}

\hypertarget{resources-and-budget}{%
\subsection{5. Resources and budget}\label{resources-and-budget}}

\hypertarget{the-research-team}{%
\subsubsection{The research team}\label{the-research-team}}

\begin{wrapfigure}{r}[0pt]{0.6\textwidth}
\centering
    \includegraphics[width=0.9\linewidth]{Gantt/Gantt-B1.pdf}
    \caption{Gantt diagram and task allocation}
    \label{fig:gantt-b1}
\end{wrapfigure}

The budget required for the development of SDRAM is XXX €. The most
significant cost is the personnel cost (XXXX € - XX \%). Minor cost
cover the purchase of open hardware equipement (XXXX € - XX \%), travels
for dissemination of results (XXXX € - XX \%), Open access fees for at
least 8 publications (XXXX € - XX \%). \%

\newpage

\hypertarget{references}{%
\subsection*{References}\label{references}}
\addcontentsline{toc}{subsection}{References}

\hypertarget{refs}{}
\begin{CSLReferences}{0}{0}
\leavevmode\vadjust pre{\hypertarget{ref-de-la-torre2021}{}}%
\CSLLeftMargin{1. }%
\CSLRightInline{De-la-Torre GE, Dioses-Salinas DC, Pizarro-Ortega CI, et
al. \href{https://doi.org/10.1016/j.scitotenv.2020.142216}{New plastic
formations in the {Anthropocene}}. \emph{Science of The Total
Environment} 2021; 754: 142216.}

\leavevmode\vadjust pre{\hypertarget{ref-stoddard2021}{}}%
\CSLLeftMargin{2. }%
\CSLRightInline{Stoddard I, Anderson K, Capstick S, et al.
\href{https://doi.org/10.1146/ANNUREV-ENVIRON-012220-011104}{Three
{Decades} of {Climate Mitigation}: {Why Haven}'t {We Bent} the {Global
Emissions Curve}?}
\emph{https://doiorg/101146/annurev-environ-012220-011104} 2021; 46:
653--689.}

\leavevmode\vadjust pre{\hypertarget{ref-hermoso2022}{}}%
\CSLLeftMargin{3. }%
\CSLRightInline{Hermoso V, Carvalho SB, Giakoumi S, et al.
\href{https://doi.org/10.1016/J.ENVSCI.2021.10.028}{The {EU Biodiversity
Strategy} for 2030: {Opportunities} and challenges on the path towards
biodiversity recovery}. \emph{Environmental Science \& Policy} 2022;
127: 263--271.}

\leavevmode\vadjust pre{\hypertarget{ref-steffen2018}{}}%
\CSLLeftMargin{4. }%
\CSLRightInline{Steffen W, Rockström J, Richardson K, et al.
\href{https://doi.org/10.1073/pnas.1810141115}{Trajectories of the
{Earth System} in the {Anthropocene}}. \emph{Proceedings of the National
Academy of Sciences} 2018; 115: 8252--8259.}

\leavevmode\vadjust pre{\hypertarget{ref-steffen2011}{}}%
\CSLLeftMargin{5. }%
\CSLRightInline{Steffen W, Grinevald J, Crutzen P, et al.
\href{https://doi.org/10.1098/rsta.2010.0327}{The {Anthropocene}:
Conceptual and historical perspectives}. \emph{Philosophical
Transactions of the Royal Society A: Mathematical, Physical and
Engineering Sciences} 2011; 369: 842--867.}

\leavevmode\vadjust pre{\hypertarget{ref-ONeill2018}{}}%
\CSLLeftMargin{6. }%
\CSLRightInline{O'Neill DW, Fanning AL, Lamb WF, et al.
\href{https://doi.org/10.1038/s41893-018-0021-4}{A good life for all
within planetary boundaries}. \emph{Nature Sustainability} 2018; 1:
88--95.}

\leavevmode\vadjust pre{\hypertarget{ref-raworth2017}{}}%
\CSLLeftMargin{7. }%
\CSLRightInline{Raworth K.
\href{https://doi.org/10.1016/S2542-5196(17)30028-1}{A {Doughnut} for
the {Anthropocene}: Humanity's compass in the 21st century}. \emph{The
Lancet Planetary Health} 2017; 1: e48--e49.}

\leavevmode\vadjust pre{\hypertarget{ref-Rockstrom2009}{}}%
\CSLLeftMargin{8. }%
\CSLRightInline{Rockström J, Steffen W, Noone K, et al.
\href{https://doi.org/10.1038/461472a}{A safe operating space for
humanity}. \emph{Nature} 2009; 461: 472--475.}

\leavevmode\vadjust pre{\hypertarget{ref-kanger2022}{}}%
\CSLLeftMargin{9. }%
\CSLRightInline{Kanger L, Bone F, Rotolo D, et al.
\href{https://doi.org/10.1016/j.techfore.2022.121491}{Deep transitions:
{A} mixed methods study of the historical evolution of mass production}.
\emph{Technological Forecasting and Social Change} 2022; 177: 121491.}

\leavevmode\vadjust pre{\hypertarget{ref-EC2015}{}}%
\CSLLeftMargin{10. }%
\CSLRightInline{European Commision.
\href{https://doi.org/10.1017/CBO9781107415324.004}{Summary for
{Policymakers}}. In: Intergovernmental Panel on Climate Change (ed)
\emph{Climate {Change} 2013 - {The Physical Science Basis}}.
{Cambridge}: {Cambridge University Press}, pp. 1--30.}

\leavevmode\vadjust pre{\hypertarget{ref-EllenMacArthurFoundation2015}{}}%
\CSLLeftMargin{11. }%
\CSLRightInline{Ellen MacArthur Foundation.
\href{https://doi.org/Article}{Growth within: A circular economy vision
for a competitive europe}. \emph{Ellen MacArthur Foundation} 2015; 100.}

\leavevmode\vadjust pre{\hypertarget{ref-nobre2021}{}}%
\CSLLeftMargin{12. }%
\CSLRightInline{Nobre GC, Tavares E.
\href{https://doi.org/10.1016/j.jclepro.2021.127973}{The quest for a
circular economy final definition: {A} scientific perspective}.
\emph{Journal of Cleaner Production} 2021; 314: 127973.}

\leavevmode\vadjust pre{\hypertarget{ref-Kirchherr2017}{}}%
\CSLLeftMargin{13. }%
\CSLRightInline{Kirchherr J, Reike D, Hekkert M.
\href{https://doi.org/10.1016/j.resconrec.2017.09.005}{Conceptualizing
the circular economy: {An} analysis of 114 definitions}.
\emph{Resources, Conservation and Recycling} 2017; 127: 221--232.}

\leavevmode\vadjust pre{\hypertarget{ref-Schoggl2020}{}}%
\CSLLeftMargin{14. }%
\CSLRightInline{Schöggl J-P, Stumpf L, Baumgartner RJ.
\href{https://doi.org/10.1016/j.resconrec.2020.105073}{The narrative of
sustainability and circular economy - {A} longitudinal review of two
decades of research}. \emph{Resources, Conservation and Recycling} 2020;
163: 105073.}

\leavevmode\vadjust pre{\hypertarget{ref-CalistoFriant2020}{}}%
\CSLLeftMargin{15. }%
\CSLRightInline{Calisto Friant M, Vermeulen WJV, Salomone R.
\href{https://doi.org/10.1016/j.resconrec.2020.104917}{A typology of
circular economy discourses: {Navigating} the diverse visions of a
contested paradigm}. \emph{Resources, Conservation and Recycling} 2020;
161: 104917.}

\leavevmode\vadjust pre{\hypertarget{ref-rodl2022}{}}%
\CSLLeftMargin{16. }%
\CSLRightInline{Rödl MB, Åhlvik T, Bergeå H, et al.
\href{https://doi.org/10.1016/J.JCLEPRO.2022.132144}{Performing the
{Circular} economy: {How} an ambiguous discourse is managed and
maintained through meetings}. \emph{Journal of Cleaner Production} 2022;
360: 132144.}

\leavevmode\vadjust pre{\hypertarget{ref-corvellec2021}{}}%
\CSLLeftMargin{17. }%
\CSLRightInline{Corvellec H, Stowell AF, Johansson N. Critiques of the
circular economy. \emph{Journal of Industrial Ecology}. Epub ahead of
print 2021. DOI:
\href{https://doi.org/10.1111/JIEC.13187}{10.1111/JIEC.13187}.}

\leavevmode\vadjust pre{\hypertarget{ref-zhen2021}{}}%
\CSLLeftMargin{18. }%
\CSLRightInline{Zhen H, Gao W, Yuan K, et al.
\href{https://doi.org/10.1016/j.ecoser.2021.101323}{Internalizing
externalities through net ecosystem service analysis\textendash{{A}}
case study of greenhouse vegetable farms in {Beijing}}. \emph{Ecosystem
Services} 2021; 50: 101323.}

\leavevmode\vadjust pre{\hypertarget{ref-touriki2021}{}}%
\CSLLeftMargin{19. }%
\CSLRightInline{Touriki FE, Benkhati I, Kamble SS, et al.
\href{https://doi.org/10.1016/J.JCLEPRO.2021.128691}{An integrated
smart, green, resilient, and lean manufacturing framework: {A}
literature review and future research directions}. \emph{Journal of
Cleaner Production} 2021; 319: 128691.}

\leavevmode\vadjust pre{\hypertarget{ref-VanFan2019}{}}%
\CSLLeftMargin{20. }%
\CSLRightInline{Van Fan Y, Lee CT, Lim JS, et al.
\href{https://doi.org/10.1016/j.jclepro.2019.05.266}{Cross-disciplinary
{Approaches Towards Smart}, {Resilient} and {Sustainable Circular
Economy}}. \emph{Journal of Cleaner Production} 2019; 232: 1482--1491.}

\leavevmode\vadjust pre{\hypertarget{ref-weichhart2021}{}}%
\CSLLeftMargin{21. }%
\CSLRightInline{Weichhart G, Mangler J, Raschendorfer A, et al.
\href{https://doi.org/10.1007/s00502-021-00912-2}{An adaptive
system-of-systems approach for resilient manufacturing}. \emph{e \& i
Elektrotechnik und Informationstechnik} 2021; 138: 341--348.}

\leavevmode\vadjust pre{\hypertarget{ref-kallis2018}{}}%
\CSLLeftMargin{22. }%
\CSLRightInline{Kallis G, Kostakis V, Lange S, et al.
\href{https://doi.org/10.1146/annurev-environ-102017-025941}{Research
{On Degrowth}}. \emph{Annual Review of Environment and Resources} 2018;
43: 291--316.}

\leavevmode\vadjust pre{\hypertarget{ref-savini2021}{}}%
\CSLLeftMargin{23. }%
\CSLRightInline{Savini F.
\href{https://doi.org/10.1080/09640568.2020.1857226}{The circular
economy of waste: Recovery, incineration and urban reuse}. \emph{Journal
of Environmental Planning and Management} 2021; 64: 2114--2132.}

\leavevmode\vadjust pre{\hypertarget{ref-Bakshi2018}{}}%
\CSLLeftMargin{24. }%
\CSLRightInline{Bakshi BR, Gutowski TG, Sekulic DP.
\href{https://doi.org/10.1021/acssuschemeng.7b03953}{Claiming
{Sustainability}: {Requirements} and {Challenges}}. \emph{ACS
Sustainable Chemistry \& Engineering} 2018; 6: 3632--3639.}

\leavevmode\vadjust pre{\hypertarget{ref-Bakshi2019a}{}}%
\CSLLeftMargin{25. }%
\CSLRightInline{Bakshi BR.
\href{https://doi.org/10.1146/annurev-chembioeng-060718-030332}{Toward
sustainable chemical engineering: {The} role of process systems
engineering}. \emph{Annual Review of Chemical and Biomolecular
Engineering} 2019; 10: 265--288.}

\leavevmode\vadjust pre{\hypertarget{ref-Zheng2020}{}}%
\CSLLeftMargin{26. }%
\CSLRightInline{Zheng C, Yuan J, Zhu L, et al.
\href{https://doi.org/10.1016/j.jclepro.2020.120689}{From digital to
sustainable: {A} scientometric review of smart city literature between
1990 and 2019}. \emph{Journal of Cleaner Production} 2020; 258: 120689.}

\leavevmode\vadjust pre{\hypertarget{ref-Sodiq2019}{}}%
\CSLLeftMargin{27. }%
\CSLRightInline{Sodiq A, Baloch AAB, Khan SA, et al.
\href{https://doi.org/10.1016/j.jclepro.2019.04.106}{Towards modern
sustainable cities: {Review} of sustainability principles and trends}.
\emph{Journal of Cleaner Production} 2019; 227: 972--1001.}

\leavevmode\vadjust pre{\hypertarget{ref-schraven2021}{}}%
\CSLLeftMargin{28. }%
\CSLRightInline{Schraven D, Joss S, de Jong M.
\href{https://doi.org/10.1016/j.jclepro.2021.125924}{Past, present,
future: {Engagement} with sustainable urban development through 35 city
labels in the scientific literature 1990\textendash 2019}. \emph{Journal
of Cleaner Production} 2021; 292: 125924.}

\leavevmode\vadjust pre{\hypertarget{ref-Riffat2016}{}}%
\CSLLeftMargin{29. }%
\CSLRightInline{Riffat S, Powell R, Aydin D.
\href{https://doi.org/10.1186/s40984-016-0014-2}{Future cities and
environmental sustainability}. \emph{Future Cities and Environment}
2016; 2: 1.}

\leavevmode\vadjust pre{\hypertarget{ref-Herrmann2020}{}}%
\CSLLeftMargin{30. }%
\CSLRightInline{Herrmann C, Juraschek M, Burggräf P, et al.
\href{https://doi.org/10.1016/j.cirp.2020.05.003}{Urban production:
{State} of the art and future trends for urban factories}. \emph{CIRP
Annals} 2020; 69: 764--787.}

\leavevmode\vadjust pre{\hypertarget{ref-juraschek2022}{}}%
\CSLLeftMargin{31. }%
\CSLRightInline{Juraschek M. \emph{Analysis and {Development} of
{Sustainable Urban Production Systems}}. {Cham}: {Springer International
Publishing}, 2022. Epub ahead of print 2022. DOI:
\href{https://doi.org/10.1007/978-3-030-76602-3}{10.1007/978-3-030-76602-3}.}

\leavevmode\vadjust pre{\hypertarget{ref-priavolou2022}{}}%
\CSLLeftMargin{32. }%
\CSLRightInline{Priavolou C, Troullaki K, Tsiouris N, et al.
\href{https://doi.org/10.1016/j.jclepro.2022.134291}{Tracing sustainable
production from a degrowth and localisation perspective: {A} case of
{3D} printers}. \emph{Journal of Cleaner Production} 2022; 376: 134291.}

\leavevmode\vadjust pre{\hypertarget{ref-cerdas2017}{}}%
\CSLLeftMargin{33. }%
\CSLRightInline{Cerdas F, Juraschek M, Thiede S, et al.
\href{https://doi.org/10.1111/jiec.12618}{Life {Cycle Assessment} of {3D
Printed Products} in a {Distributed Manufacturing System}}.
\emph{Journal of Industrial Ecology} 2017; 21: S80--S93.}

\leavevmode\vadjust pre{\hypertarget{ref-Kostakis2018}{}}%
\CSLLeftMargin{34. }%
\CSLRightInline{Kostakis V, Latoufis K, Liarokapis M, et al.
\href{https://doi.org/10.1016/j.jclepro.2016.09.077}{The convergence of
digital commons with local manufacturing from a degrowth perspective:
{Two} illustrative cases}. \emph{Journal of Cleaner Production} 2018;
197: 1684--1693.}

\leavevmode\vadjust pre{\hypertarget{ref-sabel1985}{}}%
\CSLLeftMargin{35. }%
\CSLRightInline{Sabel C, Zeitlin J. Historical {Alternatives} to {Mass
Production}: {Politics}, {Markets} and {Technology} in
{Nineteenth-Century Industrialization}. \emph{Past \& Present} 1985;
133--176.}

\leavevmode\vadjust pre{\hypertarget{ref-Pearce2010}{}}%
\CSLLeftMargin{36. }%
\CSLRightInline{Pearce JM, Morris Blair C, Laciak KJ, et al.
\href{https://doi.org/10.5539/jsd.v3n4p17}{3-{D Printing} of {Open
Source Appropriate Technologies} for {Self-Directed Sustainable
Development}}. \emph{Journal of Sustainable Development} 2010; 3:
17--29.}

\leavevmode\vadjust pre{\hypertarget{ref-Kostakis2013}{}}%
\CSLLeftMargin{37. }%
\CSLRightInline{Kostakis V, Papachristou M.
\href{https://doi.org/10.1016/j.tele.2013.09.006}{Commons-based peer
production and digital fabrication: {The} case of a {RepRap-based},
{Lego-built 3D} printing-milling machine}. \emph{Telematics and
Informatics} 2014; 31: 434--443.}

\leavevmode\vadjust pre{\hypertarget{ref-Heikkinen2020a}{}}%
\CSLLeftMargin{38. }%
\CSLRightInline{Heikkinen ITS, Savin H, Partanen J, et al.
\href{https://doi.org/10.1016/j.techfore.2020.119986}{Towards national
policy for open source hardware research: {The} case of {Finland}}.
2020; 155: 119986.}

\leavevmode\vadjust pre{\hypertarget{ref-dibona1999}{}}%
\CSLLeftMargin{39. }%
\CSLRightInline{DiBona C, Ockman S. \emph{Open sources: Voices from the
open source revolution}. " O'Reilly Media, Inc.", 1999.}

\leavevmode\vadjust pre{\hypertarget{ref-raymond1999}{}}%
\CSLLeftMargin{40. }%
\CSLRightInline{Raymond E. The cathedral and the bazaar.
\emph{Knowledge, Technology \& Policy} 1999; 12: 23--49.}

\leavevmode\vadjust pre{\hypertarget{ref-lakhani2004}{}}%
\CSLLeftMargin{41. }%
\CSLRightInline{Lakhani KR, Von Hippel E. How open source software
works:{`free'} user-to-user assistance. In: \emph{Produktentwicklung mit
virtuellen communities}. Springer, 2004, pp. 303--339.}

\leavevmode\vadjust pre{\hypertarget{ref-deek2007}{}}%
\CSLLeftMargin{42. }%
\CSLRightInline{Deek FP, McHugh JA. \emph{Open source: Technology and
policy}. Cambridge University Press, 2007.}

\leavevmode\vadjust pre{\hypertarget{ref-colombo2014}{}}%
\CSLLeftMargin{43. }%
\CSLRightInline{Colombo MG, Piva E, Rossi-Lamastra C. Open innovation
and within-industry diversification in small and medium enterprises: The
case of open source software firms. \emph{Research Policy} 2014; 43:
891--902.}

\leavevmode\vadjust pre{\hypertarget{ref-dodourova2014}{}}%
\CSLLeftMargin{44. }%
\CSLRightInline{Dodourova M, Bevis K. Networking innovation in the
european car industry: Does the open innovation model fit?
\emph{Transportation Research Part A: Policy and Practice} 2014; 69:
252--271.}

\leavevmode\vadjust pre{\hypertarget{ref-alexy2013}{}}%
\CSLLeftMargin{45. }%
\CSLRightInline{Alexy O, Henkel J, Wallin MW. From closed to open: Job
role changes, individual predispositions, and the adoption of commercial
open source software development. \emph{Research Policy} 2013; 42:
1325--1340.}

\leavevmode\vadjust pre{\hypertarget{ref-boudreau2016}{}}%
\CSLLeftMargin{46. }%
\CSLRightInline{Boudreau KJ, Lakhani KR. Innovation experiments:
Researching technical advance, knowledge production, and the design of
supporting institutions. \emph{Innovation Policy and the Economy} 2016;
16: 135--167.}

\leavevmode\vadjust pre{\hypertarget{ref-hienerth2014}{}}%
\CSLLeftMargin{47. }%
\CSLRightInline{Hienerth C, Von Hippel E, Jensen MB. User community vs.
Producer innovation development efficiency: A first empirical study.
\emph{Research policy} 2014; 43: 190--201.}

\leavevmode\vadjust pre{\hypertarget{ref-soderberg2015}{}}%
\CSLLeftMargin{48. }%
\CSLRightInline{Söderberg J. \emph{Hacking capitalism: The free and open
source software movement}. Routledge, 2015.}

\leavevmode\vadjust pre{\hypertarget{ref-martinez2015}{}}%
\CSLLeftMargin{49. }%
\CSLRightInline{Martinez MG. Solver engagement in knowledge sharing in
crowdsourcing communities: Exploring the link to creativity.
\emph{Research Policy} 2015; 44: 1419--1430.}

\leavevmode\vadjust pre{\hypertarget{ref-Ardal2016}{}}%
\CSLLeftMargin{50. }%
\CSLRightInline{Årdal C, Røttingen JA. {Financing and collaboration on
research and development for nodding syndrome}. \emph{Health Research
Policy and Systems} 2016; 14: 1--7.}

\leavevmode\vadjust pre{\hypertarget{ref-thompson2011}{}}%
\CSLLeftMargin{51. }%
\CSLRightInline{Thompson C. Build it. Share it. Profit. Can open source
hardware work. \emph{Work}; 10.}

\leavevmode\vadjust pre{\hypertarget{ref-fisher2012}{}}%
\CSLLeftMargin{52. }%
\CSLRightInline{Fisher DK, Gould PJ. Open-source hardware is a low-cost
alternative for scientific instrumentation and research. \emph{Modern
instrumentation} 2012; 1: 8.}

\leavevmode\vadjust pre{\hypertarget{ref-pearce2012}{}}%
\CSLLeftMargin{53. }%
\CSLRightInline{Pearce JM. Building research equipment with free,
open-source hardware. \emph{Science} 2012; 337: 1303--1304.}

\leavevmode\vadjust pre{\hypertarget{ref-pearce2013}{}}%
\CSLLeftMargin{54. }%
\CSLRightInline{Pearce JM. \emph{Open-source lab: How to build your own
hardware and reduce research costs}. Newnes, 2013.}

\leavevmode\vadjust pre{\hypertarget{ref-li2018}{}}%
\CSLLeftMargin{55. }%
\CSLRightInline{Li Z, Seering W, Wallace D. Understanding value
propositions and revenue models in open source hardware companies. In:
\emph{ICIE 2018 6th international conference on innovation and
entrepreneurship: ICIE 2018}. Academic Conferences; publishing limited,
2018, p. 214.}

\leavevmode\vadjust pre{\hypertarget{ref-pearce2018}{}}%
\CSLLeftMargin{56. }%
\CSLRightInline{Pearce J. Sponsored libre research agreements to create
free and open source software and hardware. \emph{Inventions} 2018; 3:
44.}

\leavevmode\vadjust pre{\hypertarget{ref-petch2014}{}}%
\CSLLeftMargin{57. }%
\CSLRightInline{Petch A, Lightowler C, Pattoni L, et al. Embedding
research into practice through innovation and creativity: A case study
from social services. \emph{Evidence \& Policy: A Journal of Research,
Debate and Practice} 2014; 10: 555--564.}

\leavevmode\vadjust pre{\hypertarget{ref-pearce2015a}{}}%
\CSLLeftMargin{58. }%
\CSLRightInline{Pearce JM. Return on investment for open source
scientific hardware development. \emph{Science and Public Policy} 2015;
43: 192--195.}

\leavevmode\vadjust pre{\hypertarget{ref-pearce2015b}{}}%
\CSLLeftMargin{59. }%
\CSLRightInline{Pearce JM. Quantifying the value of open source hardware
development. \emph{Modern Economy} 2015; 6: 1--11.}

\leavevmode\vadjust pre{\hypertarget{ref-wittbrodt2013}{}}%
\CSLLeftMargin{60. }%
\CSLRightInline{Wittbrodt BT, Glover A, Laureto J, et al. Life-cycle
economic analysis of distributed manufacturing with open-source 3-d
printers. \emph{Mechatronics} 2013; 23: 713--726.}

\leavevmode\vadjust pre{\hypertarget{ref-Beltagui2020}{}}%
\CSLLeftMargin{61. }%
\CSLRightInline{Beltagui A, Sesis A, Stylos N.
\href{https://doi.org/10.1016/j.techfore.2020.120453}{A bricolage
perspective on democratising innovation: {The} case of {3D} printing in
makerspaces}. \emph{Technological Forecasting and Social Change} 2021;
163: 120453.}

\leavevmode\vadjust pre{\hypertarget{ref-Pearce2020a}{}}%
\CSLLeftMargin{62. }%
\CSLRightInline{Pearce JM. A review of open source ventilators for
{COVID-19} and future pandemics. \emph{F1000Research}; 9. Epub ahead of
print 2020. DOI:
\href{https://doi.org/10.12688/f1000research.22942.2}{10.12688/f1000research.22942.2}.}

\leavevmode\vadjust pre{\hypertarget{ref-He2014}{}}%
\CSLLeftMargin{63. }%
\CSLRightInline{He Y, Xue G, Fu J.
\href{https://doi.org/10.1038/srep06973}{Fabrication of low cost soft
tissue prostheses with the desktop {3D} printer}. \emph{Scientific
Reports} 2014; 4: 6973.}

\leavevmode\vadjust pre{\hypertarget{ref-tan2021}{}}%
\CSLLeftMargin{64. }%
\CSLRightInline{Tan HW, Choong YYC.
\href{https://doi.org/10.1080/17452759.2021.1975882}{Additive
manufacturing in {COVID-19}: Recognising the challenges and driving for
assurance}. \emph{https://doiorg/101080/1745275920211975882} 2021;
1--6.}

\leavevmode\vadjust pre{\hypertarget{ref-Ijassi2022}{}}%
\CSLLeftMargin{65. }%
\CSLRightInline{Ijassi W, Evrard D, Zwolinski P.
\href{https://doi.org/10.1016/j.procir.2022.02.048}{Characterizing urban
factories by their value chain: A first step towards more sustainability
in production}. \emph{Procedia CIRP} 2022; 105: 290--295.}

\leavevmode\vadjust pre{\hypertarget{ref-CruzSanchez2014}{}}%
\CSLLeftMargin{66. }%
\CSLRightInline{Cruz Sanchez FA, Boudaoud H, Muller L, et al.
\href{https://doi.org/10.1080/17452759.2014.919553}{Towards a standard
experimental protocol for open source additive manufacturing}.
\emph{Virtual and Physical Prototyping} 2014; 9: 151--167.}

\leavevmode\vadjust pre{\hypertarget{ref-Arthur2020}{}}%
\CSLLeftMargin{67. }%
\CSLRightInline{Alexandre A, Cruz Sanchez FA, Boudaoud H, et al.
\href{https://doi.org/10.1089/3dp.2019.0195}{Mechanical {Properties} of
{Direct Waste Printing} of {Polylactic Acid} with {Universal Pellets
Extruder}: {Comparison} to {Fused Filament Fabrication} on {Open-Source
Desktop Three-Dimensional Printers}}. \emph{3D Printing and Additive
Manufacturing} 2020; 3dp.2019.0195.}

\leavevmode\vadjust pre{\hypertarget{ref-Cruz2015}{}}%
\CSLLeftMargin{68. }%
\CSLRightInline{Cruz F, Lanza S, Boudaoud H, et al. Polymer {Recycling}
and {Additive Manufacturing} in an {Open Source} context :
{Optimization} of processes and methods. In: \emph{Solid {Freeform
Fabrication}}. {Austin, Texas}, 2015, pp. 1591--1600.}

\leavevmode\vadjust pre{\hypertarget{ref-CruzSanchez2017}{}}%
\CSLLeftMargin{69. }%
\CSLRightInline{Cruz Sanchez FA, Boudaoud H, Hoppe S, et al.
\href{https://doi.org/10.1016/j.addma.2017.05.013}{Polymer recycling in
an open-source additive manufacturing context: {Mechanical} issues}.
\emph{Additive Manufacturing} 2017; 17: 87--105.}

\leavevmode\vadjust pre{\hypertarget{ref-lopez2022}{}}%
\CSLLeftMargin{70. }%
\CSLRightInline{López VM, Carou D, Cruz S FA.
\href{https://doi.org/10.1177/09544054221113378}{Feasibility study on
the use of recycled materials for prototyping purposes: {A} comparative
study based on the tensile strength}. \emph{Proceedings of the
Institution of Mechanical Engineers, Part B: Journal of Engineering
Manufacture} 2022; 09544054221113378.}

\leavevmode\vadjust pre{\hypertarget{ref-Pavlo2018}{}}%
\CSLLeftMargin{71. }%
\CSLRightInline{Pavlo S, Fabio C, Hakim B, et al.
\href{https://doi.org/10.1109/ICE.2018.8436296}{{3D-Printing Based
Distributed Plastic Recycling}: {A Conceptual Model} for {Closed-Loop
Supply Chain Design}}. In: \emph{2018 {IEEE International Conference} on
{Engineering}, {Technology} and {Innovation} ({ICE}/{ITMC})}. {IEEE},
2018, pp. 1--8.}

\leavevmode\vadjust pre{\hypertarget{ref-Santander2020}{}}%
\CSLLeftMargin{72. }%
\CSLRightInline{Santander P, Cruz Sanchez FA, Boudaoud H, et al.
\href{https://doi.org/10.1016/j.resconrec.2019.104531}{{Closed loop
supply chain network for local and distributed plastic recycling for 3D
printing: a MILP-based optimization approach}}. \emph{Resources,
Conservation and Recycling} 2020; 154: 104531.}

\leavevmode\vadjust pre{\hypertarget{ref-Santander2022}{}}%
\CSLLeftMargin{73. }%
\CSLRightInline{Santander P, Cruz Sanchez FA, Boudaoud H, et al.
\href{https://doi.org/10.1016/j.clet.2022.100397}{Social, political, and
technological dimensions of the sustainability evaluation of a recycling
network. {A} literature review}. \emph{Cleaner Engineering and
Technology} 2022; 6: 100397.}

\leavevmode\vadjust pre{\hypertarget{ref-CruzSanchez2020}{}}%
\CSLLeftMargin{74. }%
\CSLRightInline{Cruz Sanchez FA, Boudaoud H, Camargo M, et al.
\href{https://doi.org/10.1016/j.jclepro.2020.121602}{Plastic recycling
in additive manufacturing: {A} systematic literature review and
opportunities for the circular economy}. \emph{Journal of Cleaner
Production} 2020; 264: 121602.}

\leavevmode\vadjust pre{\hypertarget{ref-Baechler2013}{}}%
\CSLLeftMargin{75. }%
\CSLRightInline{Baechler C, DeVuono M, Pearce JM. {Distributed recycling
of waste polymer into RepRap feedstock}. \emph{Rapid Prototyping
Journal} 2013; 19: 118--125.}

\leavevmode\vadjust pre{\hypertarget{ref-JustinoNetto2021}{}}%
\CSLLeftMargin{76. }%
\CSLRightInline{Justino Netto JM, Idogava HT, Frezzatto Santos LE, et
al. \href{https://doi.org/10.1007/s00170-021-07365-z}{Screw-assisted
{3D} printing with granulated materials: A systematic review}. \emph{The
International Journal of Advanced Manufacturing Technology} 2021;
1--17.}

\leavevmode\vadjust pre{\hypertarget{ref-netto2022}{}}%
\CSLLeftMargin{77. }%
\CSLRightInline{Netto JMJ, Sarout AI, Santos ALG, et al.
\href{https://doi.org/10.1016/j.addma.2022.103192}{{DESIGN AND
VALIDATION OF AN INNOVATIVE 3D PRINTER CONTAINING A CO-ROTATING TWIN
SCREW EXTRUSION UNIT}}. \emph{Additive Manufacturing} 2022; 103192.}

\leavevmode\vadjust pre{\hypertarget{ref-billah2021}{}}%
\CSLLeftMargin{78. }%
\CSLRightInline{Billah KMM, Heineman J, Mhatre P, et al.
\href{https://doi.org/10.1016/J.ADDMA.2021.102282}{Large-scale additive
manufacturing of self-heating molds}. \emph{Additive Manufacturing}
2021; 47: 102282.}

\leavevmode\vadjust pre{\hypertarget{ref-Reich2019}{}}%
\CSLLeftMargin{79. }%
\CSLRightInline{Reich MJ, Woern AL, Tanikella NG, et al.
\href{https://doi.org/10.3390/ma12101642}{Mechanical {Properties} and
{Applications} of {Recycled Polycarbonate Particle Material
Extrusion-Based Additive Manufacturing}}. \emph{Materials} 2019; 12:
1642.}

\leavevmode\vadjust pre{\hypertarget{ref-Byard2019}{}}%
\CSLLeftMargin{80. }%
\CSLRightInline{Byard DJ, Woern AL, Oakley RB, et al.
\href{https://doi.org/10.1016/j.addma.2019.03.006}{Green fab lab
applications of large-area waste polymer-based additive manufacturing}.
\emph{Additive Manufacturing} 2019; 27: 515--525.}

\leavevmode\vadjust pre{\hypertarget{ref-petsiuk2022}{}}%
\CSLLeftMargin{81. }%
\CSLRightInline{Petsiuk A, Lavu B, Dick R, et al.
\href{https://doi.org/10.3390/inventions7030070}{Waste {Plastic Direct
Extrusion Hangprinter}}. \emph{Inventions} 2022; 7: 70.}

\leavevmode\vadjust pre{\hypertarget{ref-Zhong2018}{}}%
\CSLLeftMargin{82. }%
\CSLRightInline{Zhong S, Pearce JM.
\href{https://doi.org/10.1016/j.resconrec.2017.09.023}{Tightening the
loop on the circular economy: {Coupled} distributed recycling and
manufacturing with recyclebot and {RepRap} 3-{D} printing}.
\emph{Resources, Conservation and Recycling} 2018; 128: 48--58.}

\leavevmode\vadjust pre{\hypertarget{ref-Garmulewicz2018}{}}%
\CSLLeftMargin{83. }%
\CSLRightInline{Garmulewicz A, Holweg M, Veldhuis H, et al.
\href{https://doi.org/10.1177/0008125617752695}{Disruptive {Technology}
as an {Enabler} of the {Circular Economy}: {What Potential Does 3D
Printing Hold}?} \emph{California Management Review} 2018; 60:
112--132.}

\leavevmode\vadjust pre{\hypertarget{ref-Despeisse2016}{}}%
\CSLLeftMargin{84. }%
\CSLRightInline{Despeisse M, Baumers M, Brown P, et al.
\href{https://doi.org/10.1016/j.techfore.2016.09.021}{Unlocking value
for a circular economy through {3D} printing: {A} research agenda}.
\emph{Technological Forecasting and Social Change} 2017; 115: 75--84.}

\leavevmode\vadjust pre{\hypertarget{ref-Kreiger2013}{}}%
\CSLLeftMargin{85. }%
\CSLRightInline{Kreiger M, Pearce JM.
\href{https://doi.org/10.1557/opl.2013.319}{Environmental {Impacts} of
{Distributed Manufacturing} from 3-{D Printing} of {Polymer Components}
and {Products}}. \emph{MRS Proceedings} 2013; 1492: 85--90.}

\leavevmode\vadjust pre{\hypertarget{ref-Zhong2017}{}}%
\CSLLeftMargin{86. }%
\CSLRightInline{Zhong S, Rakhe P, Pearce J.
\href{https://doi.org/10.3390/recycling2020010}{Energy {Payback Time} of
a {Solar Photovoltaic Powered Waste Plastic Recyclebot System}}.
\emph{Recycling} 2017; 2: 10.}

\leavevmode\vadjust pre{\hypertarget{ref-Horta2017}{}}%
\CSLLeftMargin{87. }%
\CSLRightInline{Horta JF, Simões FJP, Mateus A.
\href{https://doi.org/10.1007/s12289-017-1364-5}{Large scale additive
manufacturing of eco-composites}. \emph{International Journal of
Material Forming} 2018; 11: 375--380.}

\leavevmode\vadjust pre{\hypertarget{ref-tyl2021}{}}%
\CSLLeftMargin{88. }%
\CSLRightInline{Tyl B, Allais R.
\href{https://doi.org/10.1016/J.JCLEPRO.2021.129032}{A design study into
multi-level living labs for reuse and repair activities in {France}}.
\emph{Journal of Cleaner Production} 2021; 321: 129032.}

\leavevmode\vadjust pre{\hypertarget{ref-compagnucci2020a}{}}%
\CSLLeftMargin{89. }%
\CSLRightInline{Compagnucci L, Spigarelli F, Coelho J, et al.
\href{https://doi.org/10.1016/j.jclepro.2020.125721}{Living {Labs} and
{User Engagement} for {Innovation} and {Sustainability}}. \emph{Journal
of Cleaner Production} 2020; 125721.}

\leavevmode\vadjust pre{\hypertarget{ref-Ceschin2016}{}}%
\CSLLeftMargin{90. }%
\CSLRightInline{Ceschin F, Gaziulusoy I.
\href{https://doi.org/10.1016/j.destud.2016.09.002}{Evolution of design
for sustainability: {From} product design to design for system
innovations and transitions}. \emph{Design Studies} 2016; 47: 118--163.}

\leavevmode\vadjust pre{\hypertarget{ref-wang2019b}{}}%
\CSLLeftMargin{91. }%
\CSLRightInline{Wang M, Liu P, Gu Z, et al.
\href{https://doi.org/10.3390/ijerph16234654}{A {Scientometric Review}
of {Resource Recycling Industry}}. \emph{International Journal of
Environmental Research and Public Health} 2019; 16: 4654.}

\leavevmode\vadjust pre{\hypertarget{ref-leipold2021}{}}%
\CSLLeftMargin{92. }%
\CSLRightInline{Leipold S, Weldner K, Hohl M.
\href{https://doi.org/10.1016/j.ecolecon.2021.107086}{Do we need a
{`circular society'}? {Competing} narratives of the circular economy in
the {French} food sector}. \emph{Ecological Economics} 2021; 187:
107086.}

\leavevmode\vadjust pre{\hypertarget{ref-hobson2021}{}}%
\CSLLeftMargin{93. }%
\CSLRightInline{Hobson K, Holmes H, Welch D, et al.
\href{https://doi.org/10.1016/J.JCLEPRO.2021.128969}{Consumption {Work}
in the circular economy: {A} research agenda.} \emph{Journal of Cleaner
Production} 2021; 321: 128969.}

\leavevmode\vadjust pre{\hypertarget{ref-jaeger-erben2021a}{}}%
\CSLLeftMargin{94. }%
\CSLRightInline{Jaeger-Erben M, Jensen C, Hofmann F, et al.
\href{https://doi.org/10.1016/j.resconrec.2021.105476}{There is no
sustainable circular economy without a circular society}.
\emph{Resources, Conservation and Recycling} 2021; 168: 105476.}

\leavevmode\vadjust pre{\hypertarget{ref-klotz2022}{}}%
\CSLLeftMargin{95. }%
\CSLRightInline{Klotz M, Haupt M, Hellweg S.
\href{https://doi.org/10.1016/J.WASMAN.2022.01.002}{Limited utilization
options for secondary plastics may restrict their circularity}.
\emph{Waste Management} 2022; 141: 251--270.}

\leavevmode\vadjust pre{\hypertarget{ref-hultman2021}{}}%
\CSLLeftMargin{96. }%
\CSLRightInline{Hultman J, Corvellec H, Jerneck A, et al.
\href{https://doi.org/10.1016/j.respol.2021.104297}{A resourcification
manifesto: {Understanding} the social process of resources becoming
resources}. \emph{Research Policy} 2021; 50: 104297.}

\leavevmode\vadjust pre{\hypertarget{ref-Bakshi2015}{}}%
\CSLLeftMargin{97. }%
\CSLRightInline{Bakshi BR, Ziv G, Lepech MD.
\href{https://doi.org/10.1021/es5041442}{Techno-{Ecological Synergy}: {A
Framework} for {Sustainable Engineering}}. \emph{Environmental Science
\& Technology} 2015; 49: 1752--1760.}

\leavevmode\vadjust pre{\hypertarget{ref-Saladini2018}{}}%
\CSLLeftMargin{98. }%
\CSLRightInline{Saladini F, Gopalakrishnan V, Bastianoni S, et al.
\href{https://doi.org/10.1016/j.ecoser.2018.02.004}{Synergies between
industry and nature \textendash{} {An} emergy evaluation of a biodiesel
production system integrated with ecological systems}. \emph{Ecosystem
Services} 2018; 30: 257--266.}

\leavevmode\vadjust pre{\hypertarget{ref-xu2021e}{}}%
\CSLLeftMargin{99. }%
\CSLRightInline{Xu X, Wang L, Fratini L, et al.
\href{https://doi.org/10.1016/j.jmsy.2021.07.025}{Smart and resilient
manufacturing in the wake of {COVID-19}}. \emph{Journal of Manufacturing
Systems} 2021; 60: 707--708.}

\leavevmode\vadjust pre{\hypertarget{ref-ritala2022}{}}%
\CSLLeftMargin{100. }%
\CSLRightInline{Ritala P, De Kort C, Gailly B.
\href{https://doi.org/10.1177/01492063221086247}{Orchestrating
{Knowledge Networks}: {Alter-Oriented Brokering}}. \emph{Journal of
Management} 2022; 01492063221086247.}

\leavevmode\vadjust pre{\hypertarget{ref-Tsui2020}{}}%
\CSLLeftMargin{101. }%
\CSLRightInline{Tsui T, Peck D, Geldermans B, et al.
\href{https://doi.org/10.3390/su13010023}{The role of urban
manufacturing for a circular economy in cities}. \emph{Sustainability
(Switzerland)} 2021; 13: 1--22.}

\leavevmode\vadjust pre{\hypertarget{ref-herrmann2019}{}}%
\CSLLeftMargin{102. }%
\CSLRightInline{Herrmann C, Juraschek M, Kara S, et al.
\href{https://doi.org/10.1007/978-981-13-1181-9_15}{Urban {Factories}:
{Identifying Products} for {Production} in {Cities}}. In: Hu AH,
Matsumoto M, Kuo TC, et al. (eds) \emph{Technologies and
{Eco-innovation} towards {Sustainability I}: {Eco Design} of {Products}
and {Services}}. {Singapore}: {Springer}, 2019, pp. 185--198.}

\leavevmode\vadjust pre{\hypertarget{ref-williams2019}{}}%
\CSLLeftMargin{103. }%
\CSLRightInline{Williams J.
\href{https://doi.org/10.3390/su11020423}{Circular {Cities}:
{Challenges} to {Implementing Looping Actions}}. \emph{Sustainability}
2019; 11: 423.}

\leavevmode\vadjust pre{\hypertarget{ref-Shabbir2021}{}}%
\CSLLeftMargin{104. }%
\CSLRightInline{Shabbir MS, Mahmood A, Setiawan R, et al.
\href{https://doi.org/10.1007/s11356-021-12980-0}{Closed-loop supply
chain network design with sustainability and resiliency criteria}.
\emph{Environmental Science and Pollution Research} 2021; 1--16.}

\leavevmode\vadjust pre{\hypertarget{ref-mubarik2021}{}}%
\CSLLeftMargin{105. }%
\CSLRightInline{Mubarik MS, Naghavi N, Mubarik M, et al.
\href{https://doi.org/10.1016/j.jclepro.2021.126058}{Resilience and
cleaner production in industry 4.0: {Role} of supply chain mapping and
visibility}. \emph{Journal of Cleaner Production} 2021; 292: 126058.}

\leavevmode\vadjust pre{\hypertarget{ref-Pearce2012b}{}}%
\CSLLeftMargin{106. }%
\CSLRightInline{Pearce JM.
\href{https://doi.org/10.1007/s10668-012-9337-9}{The case for open
source appropriate technology}. \emph{Environment, Development and
Sustainability} 2012; 14: 425--431.}

\leavevmode\vadjust pre{\hypertarget{ref-Pearce2014k}{}}%
\CSLLeftMargin{107. }%
\CSLRightInline{Pearce JM.
\href{https://doi.org/10.1016/B978-0-12-410462-4.00005-6}{Open-{Source
Lab}}. In: \emph{Open-{Source Lab}}. {Elsevier}, 2014, pp. 95--162.}

\leavevmode\vadjust pre{\hypertarget{ref-Pearce2016}{}}%
\CSLLeftMargin{108. }%
\CSLRightInline{Pearce JM.
\href{https://doi.org/10.1093/scipol/scv034}{Return on investment for
open source scientific hardware development}. \emph{Sci Public Policy}
2016; 43: 192--195.}

\leavevmode\vadjust pre{\hypertarget{ref-pearce2022a}{}}%
\CSLLeftMargin{109. }%
\CSLRightInline{Pearce JM.
\href{https://doi.org/10.3390/technologies10020053}{Strategic
{Investment} in {Open Hardware} for {National Security}}.
\emph{Technologies} 2022; 10: 53.}

\leavevmode\vadjust pre{\hypertarget{ref-kish2021}{}}%
\CSLLeftMargin{110. }%
\CSLRightInline{Kish K, Mallery D, Yahya Haage G, et al.
\href{https://doi.org/10.1016/j.ecolecon.2021.107171}{Fostering critical
pluralism with systems theory, methods, and heuristics}.
\emph{Ecological Economics} 2021; 189: 107171.}

\leavevmode\vadjust pre{\hypertarget{ref-economics2021}{}}%
\CSLLeftMargin{111. }%
\CSLRightInline{Economics E, Ee W. Ecological economics : {The} next 30
years. 190. Epub ahead of print 2021. DOI:
\href{https://doi.org/10.1016/j.ecolecon.2021.107211}{10.1016/j.ecolecon.2021.107211}.}

\leavevmode\vadjust pre{\hypertarget{ref-gunton2022}{}}%
\CSLLeftMargin{112. }%
\CSLRightInline{Gunton RM, Hejnowicz AP, Basden A, et al.
\href{https://doi.org/10.1016/j.ecolecon.2022.107420}{Valuing beyond
economics: {A} pluralistic evaluation framework for participatory
policymaking}. \emph{Ecological Economics} 2022; 196: 107420.}

\leavevmode\vadjust pre{\hypertarget{ref-kennedy2022}{}}%
\CSLLeftMargin{113. }%
\CSLRightInline{Kennedy C.
\href{https://doi.org/10.1016/J.ECOLECON.2021.107272}{The {Intersection}
of {Biophysical Economics} and {Political Economy}}. \emph{Ecological
Economics} 2022; 192: 107272.}

\leavevmode\vadjust pre{\hypertarget{ref-Martinez-Hernandez2017}{}}%
\CSLLeftMargin{114. }%
\CSLRightInline{Martinez-Hernandez E.
\href{https://doi.org/10.1016/j.coche.2017.05.005}{Trends in sustainable
process design\textemdash from molecular to global scales}.
\emph{Current Opinion in Chemical Engineering} 2017; 17: 35--41.}

\leavevmode\vadjust pre{\hypertarget{ref-kurtz2021}{}}%
\CSLLeftMargin{115. }%
\CSLRightInline{Kurtz SA.
\href{https://doi.org/10.1016/j.futures.2021.102699}{Beyond the lifetime
of organizations: {A} framework for multi-generational goal survival in
the ecology of goals}. \emph{Futures} 2021; 127: 102699.}

\leavevmode\vadjust pre{\hypertarget{ref-Abson2014}{}}%
\CSLLeftMargin{116. }%
\CSLRightInline{Abson DJ, von Wehrden H, Baumgärtner S, et al.
\href{https://doi.org/10.1016/j.ecolecon.2014.04.012}{Ecosystem services
as a boundary object for sustainability}. \emph{Ecological Economics}
2014; 103: 29--37.}

\leavevmode\vadjust pre{\hypertarget{ref-SousaRocha2019}{}}%
\CSLLeftMargin{117. }%
\CSLRightInline{Rocha CS, Antunes P, Partidário P.
\href{https://doi.org/10.1016/j.jclepro.2019.06.108}{Design for
sustainability models: {A} multiperspective review}. \emph{Journal of
Cleaner Production} 2019; 234: 1428--1445.}

\leavevmode\vadjust pre{\hypertarget{ref-Bianchi2020}{}}%
\CSLLeftMargin{118. }%
\CSLRightInline{Bianchi M, Tapia C, del Valle I.
\href{https://doi.org/10.1111/jiec.13000}{Monitoring domestic material
consumption at lower territorial levels: {A} novel data downscaling
method}. \emph{Journal of Industrial Ecology} 2020; 24: 1074--1087.}

\leavevmode\vadjust pre{\hypertarget{ref-zhang2011}{}}%
\CSLLeftMargin{119. }%
\CSLRightInline{Zhang WJ, van Luttervelt CA.
\href{https://doi.org/10.1016/j.cirp.2011.03.041}{Toward a resilient
manufacturing system}. \emph{CIRP Annals} 2011; 60: 469--472.}

\leavevmode\vadjust pre{\hypertarget{ref-reinauer2021}{}}%
\CSLLeftMargin{120. }%
\CSLRightInline{Reinauer T, Hansen UE.
\href{https://doi.org/10.1016/j.technovation.2021.102289}{Determinants
of adoption in open-source hardware: {A} review of small wind turbines}.
\emph{Technovation} 2021; 102289.}

\leavevmode\vadjust pre{\hypertarget{ref-gavras2021}{}}%
\CSLLeftMargin{121. }%
\CSLRightInline{Gavras K, Kostakis V.
\href{https://doi.org/10.1017/dsj.2021.11}{Mapping the types of
modularity in open-source hardware}. \emph{Design Science} 2021/ed; 7:
e13.}

\leavevmode\vadjust pre{\hypertarget{ref-Liu2020c}{}}%
\CSLLeftMargin{122. }%
\CSLRightInline{Liu X, Bakshi BR, Rugani B, et al. Quantification and
valuation of ecosystem services in life cycle assessment: {Application}
of the cascade framework to rice farming systems. \emph{Science of the
Total Environment}; 747. Epub ahead of print 2020. DOI:
\href{https://doi.org/10.1016/j.scitotenv.2020.141278}{10.1016/j.scitotenv.2020.141278}.}

\leavevmode\vadjust pre{\hypertarget{ref-Liu2019g}{}}%
\CSLLeftMargin{123. }%
\CSLRightInline{Liu X, Bakshi BR.
\href{https://doi.org/10.1111/jiec.12755}{Ecosystem {Services} in {Life
Cycle Assessment} while {Encouraging Techno}-{Ecological Synergies}}.
\emph{Journal of Industrial Ecology} 2019; 23: 347--360.}

\leavevmode\vadjust pre{\hypertarget{ref-giampietro2018}{}}%
\CSLLeftMargin{124. }%
\CSLRightInline{Giampietro M, Mayumi K. Unraveling the complexity of the
{Jevons Paradox}: {The} link between innovation, efficiency, and
sustainability. \emph{Frontiers in Energy Research}; 6. Epub ahead of
print April 2018. DOI:
\href{https://doi.org/10.3389/FENRG.2018.00026}{10.3389/FENRG.2018.00026}.}

\leavevmode\vadjust pre{\hypertarget{ref-Diwekar2021}{}}%
\CSLLeftMargin{125. }%
\CSLRightInline{Diwekar U, Amekudzi-Kennedy A, Bakshi B, et al.
\href{https://doi.org/10.1016/j.resconrec.2020.105140}{A perspective on
the role of uncertainty in sustainability science and engineering}.
\emph{Resources, Conservation and Recycling} 2021; 164: 105140.}

\leavevmode\vadjust pre{\hypertarget{ref-saidani2021}{}}%
\CSLLeftMargin{126. }%
\CSLRightInline{Saidani M, Yannou B, Leroy Y, et al.
\href{https://doi.org/10.1016/j.spc.2021.01.010}{Multi-tool methodology
to evaluate action levers to close the loop on critical materials
\textendash{} {Application} to precious metals used in catalytic
converters}. \emph{Sustainable Production and Consumption} 2021; 26:
999--1010.}

\leavevmode\vadjust pre{\hypertarget{ref-kuo2021}{}}%
\CSLLeftMargin{127. }%
\CSLRightInline{Kuo T-C, Hsu N-Y, Wattimena R, et al.
\href{https://doi.org/10.1016/j.jclepro.2021.126901}{Toward a circular
economy: {A} system dynamic model of recycling framework for aseptic
paper packaging waste in {Indonesia}}. \emph{Journal of Cleaner
Production} 2021; 301: 126901.}

\leavevmode\vadjust pre{\hypertarget{ref-marche2022}{}}%
\CSLLeftMargin{128. }%
\CSLRightInline{Marche B, Camargo M, Bautista Rodriguez SC, et al.
\href{https://doi.org/10.1016/j.eiar.2022.106911}{Qualitative
sustainability assessment of road verge management in {France}: {An}
approach from causal diagrams to seize the importance of impact
pathways}. \emph{Environmental Impact Assessment Review} 2022; 97:
106911.}

\leavevmode\vadjust pre{\hypertarget{ref-tomoaia-cotisel2022}{}}%
\CSLLeftMargin{129. }%
\CSLRightInline{Tomoaia-Cotisel A, Allen SD, Kim H, et al.
\href{https://doi.org/10.1002/SDR.1701}{Rigorously interpreted quotation
analysis for evaluating causal loop diagrams in late-stage
conceptualization}. \emph{System Dynamics Review} 2022; 38: 41--80.}

\leavevmode\vadjust pre{\hypertarget{ref-castro2022a}{}}%
\CSLLeftMargin{130. }%
\CSLRightInline{Castro C.
\href{https://doi.org/10.1016/j.envsci.2022.07.001}{Systems-thinking for
environmental policy coherence: {Stakeholder} knowledge, fuzzy logic,
and causal reasoning}. \emph{Environmental Science \& Policy} 2022; 136:
413--427.}

\leavevmode\vadjust pre{\hypertarget{ref-perez-perez2021}{}}%
\CSLLeftMargin{131. }%
\CSLRightInline{Pérez-Pérez JF, Parra JF, Serrano-García J.
\href{https://doi.org/10.1016/j.techsoc.2021.101579}{A system dynamics
model: {Transition} to sustainable processes}. \emph{Technology in
Society} 2021; 65: 101579.}

\leavevmode\vadjust pre{\hypertarget{ref-saidani2019}{}}%
\CSLLeftMargin{132. }%
\CSLRightInline{Saidani M, Yannou B, Leroy Y, et al.
\href{https://doi.org/10.1016/j.jclepro.2018.10.014}{A taxonomy of
circular economy indicators}. \emph{Journal of Cleaner Production} 2019;
207: 542--559.}

\leavevmode\vadjust pre{\hypertarget{ref-langley2013}{}}%
\CSLLeftMargin{133. }%
\CSLRightInline{Langley A, Smallman C, Tsoukas H, et al.
\href{https://doi.org/10.5465/amj.2013.4001}{Process {Studies} of
{Change} in {Organization} and {Management}: {Unveiling Temporality},
{Activity}, and {Flow}}. \emph{Academy of Management Journal} 2013; 56:
1--13.}

\end{CSLReferences}



\end{document}
